% !TEX program = lualatex
% !TEX encoding = UTF-8 Unicode
% !TEX spellcheck = de_DE
% 
% Vorlage für Bachelorarbeiten
% 
% Um die Vorlage mit LaTeX zu erstellen sind folgende Programme aufzurufen:
% > lualatex hauptdatei.tex
% > biber hauptdatei
% > lualatex hauptdatei.tex

\documentclass{scrbook}

% !TEX root = hauptdatei.tex
% !TEX encoding = UTF-8 Unicode
% !TEX spellcheck = de_DE
%
% für mehr Informationen zu einzelnen Paketen siehe zum Beispiel:
% http://texdoc.org/pkg/paketname
% http://ctan.org/pkg/paketname


%% Setzen von Dokumentenoptionen:
%% (aquivalent zu \documentclass[<Optionen>]{...}

%%% Layout-Einstellungen
	\KOMAoptions{ 				% aquivalent zu \documentclass[<Optionen>]{...}
		fontsize=12pt,			% Standartschriftgröße
		bibliography=totoc,	% Bibliografie soll im Inhaltsverzeichnis auftauchen
		headings=normal,		% Größe und Abstand von Überschriften
		toc=listof,				% Verzeichnisse der Gleitumgebungen ins Inhaltsverz.
		toc=indent,				% Inhaltsverzeichnis in hierarchischer Form
		listof=indent,			% andere Verzeichnisse in hierarchischer Form
		listof=totoc,           % andere verzeichnisse im Inhaltsverzeichnis führen
		twoside=false				% enseitiges Layout
	}
	\setcounter{tocdepth}{1} % Ebenentiefe des Inhaltsverzeichnis
	\usepackage{geometry, setspace}
	\geometry{
		paper=a4paper,		% DIN A4 Papier
		hmargin=30mm,			% horizontale Seitenränder
		top=15mm,				% oberer Rand
		bottom=20mm,			% unterer Rand
		includeheadfoot,	% Kopf- und Fußzeilen gehören nicht zum Rand
	}
	\onehalfspacing 		% anderthalbfacher Zeilenabstand
	% Kapitelüberschrift etwas nach oben versetzen:
	\renewcommand*{\chapterheadstartvskip}{\vspace*{-1.48\topskip}}


%% Einstellung der Schriftart:
	\usepackage{lmodern}
	% Alternativ können mit fontspec beliebige im Betriebssystem installierte Schriften verwendet werden:
	%\usepackage{fontspec}
	%\setmainfont{Constantia}
	%\setsansfont{Corbel}
	%\setmonofont{Consolas}
	% Für Serifen in den Überschriften:
	%\addtokomafont{sectioning}{\rmfamily}


%% Einstelen von Kopf- und Fußzeilen:
	\usepackage[headsepline=false]{scrlayer-scrpage}
	% automatisches Füllen der Kopfzeile mit aktuellem Kapitel/Abschnitt:
	\automark[chapter]{chapter}
	% links, mitte, rechts
	\lohead{}
	\cohead{}
	\rohead{}
	% Seitenzahl nur auf plain-Seiten im Fuß
	\lofoot{}
	\cofoot*{\pagemark}
	\rofoot{}
	% Aktivieren des festgelegten Kopfzeile:
	\pagestyle{scrheadings}


%% Sprachauswahl:
	\usepackage{polyglossia}
	\setmainlanguage{english}
	% die Sprache kann im Dokument mit
	% \begin{english} ... \end{english}
	% umgestellt werden


%% Einstellungen zu Zitaten und Bibliografie:
\usepackage{csquotes}

\usepackage[
backend=biber,
bibstyle=apa,
citestyle=apa,
]{biblatex}
\addbibresource{thesis.bib}


\setlength{\bibhang}{15pt}
\defbibenvironment{bibliography}
  {\list
     {}
     {\setlength{\leftmargin}{\bibhang}%
      \setlength{\itemindent}{-\leftmargin}%
      \setlength{\itemsep}{\bibitemsep}%
      \setlength{\parsep}{\bibparsep}}}
  {\endlist}
  {\item}
  \addspace

\setlength\bibitemsep{1.3\itemsep}


	%% sonstige Pakete:
	\usepackage{
		array,		% erweiterte Option für Tabellen
		booktabs,	% schöne Tabellen
		float,		% Platzierung von Gleitobjekten (Abb., Tab.), Eigene Gleitobjekte
		graphicx,	% ermöglicht einbinden von Grafiken mit \includegraphics
		hologo,		% für TeX-Logos
		mathtools,	% Verbesserungen für den Mathesatz (läd u.a. amsmath)
		microtype,	% mikrotypografische Verbesserungen (z.B. optischer Randausgleich)
		paralist,	% platzsparende Listen mit compactitem
		xcolor,		% Verwendung von Farbe
	}


%% LaTeX sucht nach Bildern an den hier angegebenen Stellen:
	\graphicspath{{./images/},{./}}


%% automatische PDF-Verlinkungen im Dokument:
	\usepackage[
		colorlinks=false,		% Links nicht farbig hervorheben
		pdfborder={0 0 0},		% links nicht durch PDF-Kasten hervorheben
	]{hyperref}



$if(highlighting-macros)$
$highlighting-macros$
$endif$



\addbibresource{$bibliography$}


\usepackage{tikz}

\usepackage{array,makecell}
\usepackage{caption}


\usepackage{tabularx}
    \renewcommand\tabularxcolumn[1]{m{#1}}
    \newcolumntype{C}{>{\centering\arraybackslash}X}

\newcommand{\ExternalLink}{%
    \tikz[x=1.2ex, y=1.2ex, baseline=-0.05ex]{% 
        \begin{scope}[x=1ex, y=1ex]
            \clip (-0.1,-0.1) 
                --++ (-0, 1.2) 
                --++ (0.6, 0) 
                --++ (0, -0.6) 
                --++ (0.6, 0) 
                --++ (0, -1);
            \path[draw, 
                line width = 0.5, 
                rounded corners=0.5] 
                (0,0) rectangle (1,1);
        \end{scope}
        \path[draw, line width = 0.5] (0.5, 0.5) 
            -- (1, 1);
        \path[draw, line width = 0.5] (0.6, 1) 
            -- (1, 1) -- (1, 0.6);
        }
    }


\usepackage{booktabs, caption, makecell}
\renewcommand\theadfont{\bfseries}
\usepackage{threeparttable}



%% Alle wichtigen Einstellungen sind in der Datei einstellungen.tex getätigt
%% und können dort verändert werden.

\begin{document}
	
\frontmatter
\begin{titlepage}

\begin{center}

\vspace*{1,2cm}

\huge {\bfseries $title$}\\[1.8cm]

\Large {Bachelor Thesis}\\[1cm]

\large {Department of Economics}\\[0.2cm]

\large {University of Mannheim}\\[0.5cm]

\end{center}

\vfill

\noindent submitted to:\\
$supervisor$\\[1cm]
submitted by:\\
$author$\\[1cm]
Student ID: 1731015\\
Degree Programme: Bachelor of Science in Economics (B.Sc.)\\[1cm]
Address: Meerfeldstr. 11, 68163 Mannheim\\
Phone: +49 176 95741248\\
E-Mail: benedikt.stelter@students.uni-mannheim.de\\[1cm]
Mannheim, $date$

\setcounter{page}{0}

\end{titlepage}

  \tableofcontents


%% Abbildungsverzeichnis
\listoffigures

%% Tabellenverzeichnis
\listoftables


\mainmatter

$body$


 
\backmatter


 
$if(bibliography)$
$if(biblio-title)$
$if(book-class)$
\renewcommand\bibname{$biblio-title$}
$else$
\renewcommand\refname{$biblio-title$}
$endif$
$endif$
\printbibliography$if(biblio-title)$[title=$biblio-title$]$endif$

$endif$
\chapter{Appendix}
GitHub repository: \url{https://github.com/benediktstelter/bachelor_thesis}

\noindent All of the data of the covariates and the R scripts used to analyse the data can be found in folder "scripts\_data", along with some additional information in README.



\chapter{Affidavit}
\thispagestyle{empty}

% Dieser Text entspricht den genauen Vorgaben der Richtlinien für Bachelorarbeiten sowie der Prüfungsordnung (\§ 14a) 
% Stand: November 2014
I affirm that this Bachelor thesis was written by myself without any unauthorised third-party support. All used references and resources are clearly indicated. All quotes and citations are properly referenced. This thesis was never presented in the past in the same or similar form to any examination board. 

\noindent I agree that my thesis may be subject to electronic plagiarism check. For this purpose an anonymous copy may be distributed and uploaded to
servers within and outside the University of Mannheim.

\vspace{2\baselineskip}

\noindent German translation:\\
Ich versichere, dass ich die vorliegende Arbeit ohne Hilfe Dritter und ohne Benutzung anderer
als der angegebenen Quellen und Hilfsmittel angefertigt und die den benutzten Quellen
wörtlich oder inhaltlich entnommenen Stellen als solche kenntlich gemacht habe. Diese Arbeit
hat in gleicher oder ähnlicher Form noch keiner Prüfungsbehörde vorgelegen.

\noindent Ich bin damit einverstanden, dass meine Arbeit zum Zwecke eines Plagiatsabgleichs in
elektronischer Form anonymisiert versendet und gespeichert werden kann.

\vspace{4\baselineskip}
\begin{center}
\parbox{.8\textwidth}{Mannheim, $date$ \hfill $author$}
\end{center}


 
\end{document}
