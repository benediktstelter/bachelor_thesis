% !TEX program = lualatex
% !TEX encoding = UTF-8 Unicode
% !TEX spellcheck = de_DE
% 
% Vorlage für Bachelorarbeiten
% 
% Um die Vorlage mit LaTeX zu erstellen sind folgende Programme aufzurufen:
% > lualatex hauptdatei.tex
% > biber hauptdatei
% > lualatex hauptdatei.tex

\documentclass{scrbook}

\usepackage{tikz}

\newcommand{\ExternalLink}{%
    \tikz[x=1.2ex, y=1.2ex, baseline=-0.05ex]{% 
        \begin{scope}[x=1ex, y=1ex]
            \clip (-0.1,-0.1) 
                --++ (-0, 1.2) 
                --++ (0.6, 0) 
                --++ (0, -0.6) 
                --++ (0.6, 0) 
                --++ (0, -1);
            \path[draw, 
                line width = 0.5, 
                rounded corners=0.5] 
                (0,0) rectangle (1,1);
        \end{scope}
        \path[draw, line width = 0.5] (0.5, 0.5) 
            -- (1, 1);
        \path[draw, line width = 0.5] (0.6, 1) 
            -- (1, 1) -- (1, 0.6);
        }
    }


%% Alle wichtigen Einstellungen sind in der Datei einstellungen.tex getätigt
%% und können dort verändert werden.
% !TEX root = hauptdatei.tex
% !TEX encoding = UTF-8 Unicode
% !TEX spellcheck = de_DE
%
% für mehr Informationen zu einzelnen Paketen siehe zum Beispiel:
% http://texdoc.org/pkg/paketname
% http://ctan.org/pkg/paketname


%% Setzen von Dokumentenoptionen:
%% (aquivalent zu \documentclass[<Optionen>]{...}

%%% Layout-Einstellungen
	\KOMAoptions{ 				% aquivalent zu \documentclass[<Optionen>]{...}
		fontsize=12pt,			% Standartschriftgröße
		bibliography=totoc,	% Bibliografie soll im Inhaltsverzeichnis auftauchen
		headings=normal,		% Größe und Abstand von Überschriften
		toc=listof,				% Verzeichnisse der Gleitumgebungen ins Inhaltsverz.
		toc=indent,				% Inhaltsverzeichnis in hierarchischer Form
		listof=indent,			% andere Verzeichnisse in hierarchischer Form
		listof=totoc,           % andere verzeichnisse im Inhaltsverzeichnis führen
		twoside=false				% enseitiges Layout
	}
	\setcounter{tocdepth}{1} % Ebenentiefe des Inhaltsverzeichnis
	\usepackage{geometry, setspace}
	\geometry{
		paper=a4paper,		% DIN A4 Papier
		hmargin=30mm,			% horizontale Seitenränder
		top=15mm,				% oberer Rand
		bottom=20mm,			% unterer Rand
		includeheadfoot,	% Kopf- und Fußzeilen gehören nicht zum Rand
	}
	\onehalfspacing 		% anderthalbfacher Zeilenabstand
	% Kapitelüberschrift etwas nach oben versetzen:
	\renewcommand*{\chapterheadstartvskip}{\vspace*{-1.48\topskip}}


%% Einstellung der Schriftart:
	\usepackage{lmodern}
	% Alternativ können mit fontspec beliebige im Betriebssystem installierte Schriften verwendet werden:
	%\usepackage{fontspec}
	%\setmainfont{Constantia}
	%\setsansfont{Corbel}
	%\setmonofont{Consolas}
	% Für Serifen in den Überschriften:
	%\addtokomafont{sectioning}{\rmfamily}


%% Einstelen von Kopf- und Fußzeilen:
	\usepackage[headsepline=false]{scrlayer-scrpage}
	% automatisches Füllen der Kopfzeile mit aktuellem Kapitel/Abschnitt:
	\automark[chapter]{chapter}
	% links, mitte, rechts
	\lohead{}
	\cohead{}
	\rohead{}
	% Seitenzahl nur auf plain-Seiten im Fuß
	\lofoot{}
	\cofoot*{\pagemark}
	\rofoot{}
	% Aktivieren des festgelegten Kopfzeile:
	\pagestyle{scrheadings}


%% Sprachauswahl:
	\usepackage{polyglossia}
	\setmainlanguage{english}
	% die Sprache kann im Dokument mit
	% \begin{english} ... \end{english}
	% umgestellt werden


%% Einstellungen zu Zitaten und Bibliografie:
\usepackage{csquotes}

\usepackage[
backend=biber,
bibstyle=apa,
citestyle=apa,
]{biblatex}
\addbibresource{thesis.bib}


\setlength{\bibhang}{15pt}
\defbibenvironment{bibliography}
  {\list
     {}
     {\setlength{\leftmargin}{\bibhang}%
      \setlength{\itemindent}{-\leftmargin}%
      \setlength{\itemsep}{\bibitemsep}%
      \setlength{\parsep}{\bibparsep}}}
  {\endlist}
  {\item}
  \addspace

\setlength\bibitemsep{1.3\itemsep}


	%% sonstige Pakete:
	\usepackage{
		array,		% erweiterte Option für Tabellen
		booktabs,	% schöne Tabellen
		float,		% Platzierung von Gleitobjekten (Abb., Tab.), Eigene Gleitobjekte
		graphicx,	% ermöglicht einbinden von Grafiken mit \includegraphics
		hologo,		% für TeX-Logos
		mathtools,	% Verbesserungen für den Mathesatz (läd u.a. amsmath)
		microtype,	% mikrotypografische Verbesserungen (z.B. optischer Randausgleich)
		paralist,	% platzsparende Listen mit compactitem
		xcolor,		% Verwendung von Farbe
	}


%% LaTeX sucht nach Bildern an den hier angegebenen Stellen:
	\graphicspath{{./images/},{./}}


%% automatische PDF-Verlinkungen im Dokument:
	\usepackage[
		colorlinks=false,		% Links nicht farbig hervorheben
		pdfborder={0 0 0},		% links nicht durch PDF-Kasten hervorheben
	]{hyperref}


\begin{document}
	
\frontmatter
\begin{titlepage}

\begin{center}

\vspace*{1,2cm}

\huge {\bfseries The impact of vaccine lotteries on COVID-19 vaccination
rates: Evidence from Poland}\\[1.8cm]

\Large {Bachelor Thesis}\\[1cm]

\large {Department of Economics}\\[0.2cm]

\large {University of Mannheim}\\[0.5cm]

\end{center}

\vfill

\noindent submitted to:\\
Prof.~Achim Wambach, PhD / Sabrina Schubert\\[1cm]
submitted by:\\
Benedikt Stelter\\[1cm]
Student ID: 1731015\\
Degree Programme: Bachelor of Science in Economics (B.Sc.)\\[1cm]
Address: Meerfeldstr. 11, 68163 Mannheim\\
Phone: +49 176 95741248\\
E-Mail: benedikt.stelter@students.uni-mannheim.de\\[1cm]
Mannheim, 17/03/2023

\setcounter{page}{0}

\end{titlepage}

  \tableofcontents

%% Abkürzungsverzeichnis
\chapter*{Abkürzungsverzeichnis}\label{av}
\addcontentsline{toc}{chapter}{Abkürzungsverzeichnis}
\markboth{Abkürzungsverzeichnis}{Abkürzungsverzeichnis}
\begin{tabular}{ll}
ABC & Erläuterung 1 \\
DEF & Erläuterung 2\\
GHJ & Erläuterung 3\\
\end{tabular}

%% Abbildungsverzeichnis
\listoffigures

%% Tabellenverzeichnis
\listoftables

%% Symbolverzeichnis
\chapter*{Symbolverzeichnis}\label{sv}
\addcontentsline{toc}{chapter}{Symbolverzeichnis}
\markboth{Symbolverzeichnis}{Symbolverzeichnis}
\begin{tabular}{ll}
Symbol 1 & Erläuterung 1 \\
Symbol 2 & Erläuterung 2\\
Symbol 3 & Erläuterung 3\\
\end{tabular}
 
\mainmatter

\chapter{Introduction}

\chapter{Background}

\section{Literature review}

\subsection{Incentives in public health}

Governments are facing public health challenges at every corner. But how
can they motivate people to live healthier lives? Behavioural economics
can offer several ways to influence the decisions of individuals,
probably the least severe form are nudges.

Nudging is a concept mainly brought to the public through the work of
\textcite{thaler_nudge_2008}. They define a nudge as an intervention
that ``alters people's behavior in a predictable way without forbidding
any options or significantly changing their economic incentives''
(p.~6). Nudging can be applied in various ways, it is however often
connected to lifestyle topics, such as nutrition and diet. For example,
in an experiment in Denmark \parencite{friis_comparison_2017}, the aim
was to promote the consumption of vegetables in the setting of a
self-serving buffet, which included salad and other dishes in large
bowls. As a nudge, the food environment was changed by arranging green
plants and herbs around the food bowls. In a second experiment, salad
was pre-portioned into smaller take-away bowls. The results showed that
the intake of energy from vegetables of the participants can be
increased by pre-portioning the salad. Although many studies evaluating
the impact of nudges find positive effects of the respective
interventions, these results should be interpreted with caution, since a
lot of studies, including \textcite{friis_comparison_2017}, were
conducted in the lab and may therefore not be reproducible in a
real-world setting \parencite{ledderer_nudging_2020}.

Besides nudging, policymakers could try to change economic incentives.
Governments have been using taxation as a means of influencing the
behaviour of individuals for a long time, for example in alcohol and
tobacco policy. Both in developed and developing country, it has been
shown that raising prices leads to reduced consumption of tobacco
(\cite{yeh_effects_2017}; \cite{immurana_effects_2021}).

Beyond taxation, policymakers could also use other financial incentives
to motivate changes of individual behaviour to tackle public health
challenges, such as (small) cash payments or lotteries. There have been
many studies evaluating such possible schemes, typically using
randomised control trials. Several meta-analyses found that such
incentives can be successful in inducing behaviour changes. For example,
\textcite{giles_effectiveness_2014} evaluated 16 studies on issues such
as smoking cessation, health screenings, physical activity and
vaccinations. The authors found that financial incentives are more
effective than no intervention at encouraging healthy behaviour. This
finding is also confirmed by \textcite{mantzari_personal_2015}, who
evaluated 34 studies and additionally concluded that this effect is
stronger for the most deprived individuals, thereby possibly reducing
health inequalities.

There is also some literature on lotteries as an incentive in public
health, apart from COVID-19 vaccine policy.
\textcite{bjorkman_nyqvist_incentivizing_2018} find that the
introduction of a lottery program reduced HIV incidence in Lesotho.
Lotteries can also successfully increase cycling and walking activity
(\cite{ciccone_using_2021}; \cite{patel_randomized_2018}) and
participation in chlamydia screenings \parencite{niza_vouchers_2014}.

\subsection{Incentives to increase COVID-19 vaccination rates}

In march 2020, when the COVID-19 pandemic struck the world, it was not
foreseeable that vaccines would be available in just about nine months.
When they were available and hailed as the possible end of the pandemic,
it may not have been clear that rolling out vaccines was so difficult.
In countries around the world, there was a considerable amount of
vaccine hesitancy. Dealing with this hesitancy has led policymakers and
researchers to thinking about how to increase vaccination rates, by
means of vaccination mandates and passports, nudges, cash and non-cash
prizes and lotteries.

A very widely used tool, especially in Europe and the US, were
vaccination passports. The access to social gatherings/places
(e.g.~restaurants, bars, clubs, stadiums), international travel and
quarantine regulations were subject to vaccination against COVID-19.
Besides ethical concerns, for example potential disqualification of
minorities from social life because of a historically higher distrust in
government \parencite{gostin_digital_2021}, the evidence on the
effectiveness of such regulations is mixed. A synthetic control analysis
of six countries found that COVID passports were successful in
increasing daily vaccinations in countries with lower than average
vaccination rates. In more average countries (Germany and Denmark), such
regulations were less successful \parencite{mills_effect_2022}. These
cross-country comparisons should however be treated with caution, since
there are large variations across countries when it comes to the extent
of the use of COVID passports. While a comparison of Poland (very
restricted use of COVID passport) and Lithuania (wide use of COVID
passport) suggests a positive effect of passports on vaccination rates
\parencite{walkowiak_covid-19_2021}, it has to be noted that COVID
passport may also have negative effects on vaccine uptake, since
frustration about reduced autonomy might lower willingness to get
vaccinated \parencite{porat_vaccine_2021}.

Besides the carrot-and-stick approach, governments have also used
smaller nudges and financial incentives, such as cash payments and
non-cash rewards to increase COVID-19 vaccination rates. Probably the
most influential study to date on cash payments and nudges was carried
out in Sweden \parencite{campos-mercade_monetary_2021}. Using a
randomised control trial, the authors find that even ``small'' cash
payments of around 24 US dollars can significantly increase vaccination
rates, while small nudges could not increase vaccination rates. While
some studies also suggest a measurable positive effect of cash payments
on vaccination rates (\cite{wong_guaranteed_2022};
\cite{kluver_incentives_2021}; \cite{kim_vaccination_2021-1}), there is
also evidence against this effect. In another study, the results
indicated that neither behavioural nudges (text messages) nor cash
payments could increase vaccination rates among the hesitant citizens
\parencite{jacobson_can_2022}. \textcite{sprengholz_money_2021} also
find that cash incentives do not increase the willingness to be
vaccinated.

There is also a vast literature specifically on COVID-19 vaccine
lotteries, also evaluating real-world lotteries.
\textcite{dube_exploring_2022} analyse the effectiveness of a vaccine
lottery in Québec (Canada) and find a relatively small impact on
vaccination rates. A survey in Australia found that the vaccine lottery
there was successful in increasing willingness to be vaccinated
\parencite{jun_association_2022}. The majority of studies on COVID-19
vaccine lotteries however dealt with lotteries in US states. Studies on
vaccine lotteries in Louisiana and Massachusetts found different
effects. Whereas the lottery in Louisiana increased vaccine uptake
\parencite{wang_moving_2023}, a vaccine lottery in Massachusetts did not
significantly increase vaccination rates although prizes were higher
\parencite{kim_did_2023}. There are also several papers investigating
multiple state COVID-19 lotteries at once. All of these papers find that
most, but not all of the lottery programs were successful in increasing
vaccine uptake (\cite{robertson_are_2021};
\cite{acharya_implementation_2021}; \cite{fuller_assessing_2022}).

A specific focus can be observed with respect to Ohio. There is a quite
large number of studies evaluating the ``Ohio Vax-A-Million'' lottery,
which was the first COVID-19 vaccine lottery in the US. In total, a
majority of the reviewed literature casts a positive light on the
efficacy of the lottery. While \textcite{mallow_covid-19_2022} find a
positive effect of the lottery on vaccination rates, another study
cannot support this view \parencite{walkey_lottery-based_2021}. There
have also been four studies evaluating the lottery using the SCM. These
studies construct a synthetic Ohio out of a donor pool of other US
states. Three of these studies find small positive effects of the
lottery of vaccine uptake (\cite{brehm_ohio_2022};
\cite{barber_conditional_2022}; \cite{sehgal_impact_2021}) and one does
not find a robust effect \parencite{lang_did_2022}.

So far, there has been one study dealing with the vaccine lottery
analysed in this thesis. \textcite{kuznetsova_effectiveness_2022}
shortly evaluate different vaccine incentives across Europe, including
the Polish policy. By applying an interrupted time series analysis based
on an ARIMA approach, the authors suggest a slight positive effect of
this lottery. There has however not been a study dealing in detail with
the Polish lottery yet.

\subsection{Theoretical background}

Possibly to add

\section{Institutional background}

\noindent The empirical analysis will be based on a vaccination
incentive scheme implemented in Poland from July 1, 2021 to September
30, 2021. The policy had two main elements: A lottery for all adult
vaccinated people in Poland
\parencite{service_of_the_republic_of_poland_national_2021} and a
lottery-like monetary incentive scheme for municipalities
\parencite{service_of_the_republic_of_poland_competitions_2021}. The
main price of the lottery was a cash prize of one million
zloty\footnote{At the time of announcement on 25/05/2021, this was equal to around 220 000€, with an exchange rate of around 0.22 PLN to EUR},
but it also included smaller monthly, weekly and daily cash prices and
non-cash prizes with a total volume of 140 million zloty. As part of the
monetary incentive scheme for municipalities, the municipality with the
highest percentage of the vaccinated in the country received two million
zloty. 500 other municipalities could receive 100.000 zloty each.

\chapter{Methods and data}

\section{Synthetic control method}

The synthetic control method is a relatively new method in causal
inference. It was first established by \textcite{abadie_economic_2003},
further developed by \textcite{abadie_synthetic_2010} and summarized in
\textcite{abadie_using_2021}. It has been applied widely in economics,
but also other fields such as political science. Its specific case of
application are comparative case studies (write more about CCS). For
example, it has been used to evaluate the effects of European
integration \parencite{campos_institutional_2019} or the effect of
natural disasters on economic growth
\parencite{cavallo_catastrophic_2013}. The very basic idea is to create
a synthetic control unit, to then estimate the average treatment effect
(ATE) of a policy/intervention.

\subsection*{Requirements}

There are some important requirements that need to be fulfilled, in
order to obtain a good synthetic control. The first one is that the
evaluated policy/intervention has a sufficiently large effect. When the
effect of an intervention is too small, it may not be possible to
distinguish this effect from other shocks to the outcome variable.

The second one is that there exists a suitable comparison/control group.
Countries that are also subject so similar interventions or other shocks
to the outcome variable in the given time frame should be excluded from
the donor pool. What this means for the analysis in this thesis is
outlined in section 3.1. Furthermore, we should try to select countries
that are not too different from the treated country for the donor pool,
to prevent interpolation bias(cite or write about interpolation bias).
In this analysis, the donor pool is therefore restricted to eastern
European countries.

Another important requirement is that there is no anticipation effect,
so that citizens do not anticipate the enactment of a policy. We
therefore already start our intervention at the time of announcement
(25/05/2021), since there could otherwise be an anticipation effect and
it does not make a difference at which time people are getting
vaccinated for the participation in the lottery.

Lastly, it is crucial that we do not have any spillover effects on
untreated units. This requirement should however be fulfilled in the
case of Poland's vaccine lottery. The fulfillment of these
assumptions/requirements will be further discussed in section 5.

\subsection*{Formal definition}

What we observe is some outcome variable \(Y_{jt}\) for \(J + 1\) units
from \(t=1\) to \(T\). The first unit (\(j = 1\)) is the treated unit
while all other units are untreated units. The intervention (in this
analysis the announcement of the lottery) occurs at \(T_{0}+1\), meaning
that there are \(T_{0}\) pre-intervention time periods. We also observe
\(k\) predictors, which include the pre-intervention outcome variable
and additional covariates: \(X_{1j},...,X_{kj}\). These can be
summarised by the vectors \(\mathbf{X}_{j}=(X_{1j},X_{2j},...,X_{kj})'\)
for units \(j=1,...,J + 1\). It therefore follows that the
\((k\times J)\) matrix
\(\mathbf{X}_{0}=(\mathbf{X}_{2},\mathbf{X}_{3},...,\mathbf{X}_{J + 1})\)
captures all the predictors of the untreated
units\footnote{\(\mathbf{X}_0=
\begin{bmatrix}
X_{12} & X_{13} & \dots & X_{1J+1}\\
X_{22} & X_{23} & \dots & X_{2J+1}\\
\vdots & \vdots & \ddots & \vdots\\
X_{k2} & X_{k3} & \dots & X_{kJ + 1}
\end{bmatrix}\)}.

(Potentially write about matching and assumptions). The causal effect is
defined as the difference between the potential outcome of the treated
unit with intervention, which can be defined as \(Y_{1t}^{I}\), and its
potential outcome without the intervention, \(Y_{1t}^{N}\):
\begin{equation}
\tau_{1t}=Y_{1t}^{I}-Y_{1t}^{N}\; \; \text{for}\; \; t>T_{0}
\end{equation} By definition, the outcome with intervention is known and
the outcome without intervention is hypothetical for \(t>T_{0}\) for the
treated unit. Therefore, to estimate the causal effect, it is sufficient
to estimate \(Y_{1t}^{N}\). How can this problem be solved? By
\textcite{abadie_economic_2003}, the SCM proposes to use a weighted
average of donor pool units as a synthetic control. The synthetic
control is therefore defined as: \begin{equation}
\hat{Y}_{1t}^{N}=\sum_{j=2}^{J+1} w_{j}Y_{jt}
\end{equation} The weights \(\mathbf{W}=(w_{2},w_{3},...,w_{J+1})'\) are
chosen, such that the synthetic control matches as closely as possible
the pre-intervention path of the outcome variable for the treated unit.
Therefore the weights have to be chosen so that they minimize this
difference. The optimal weights
\(\mathbf{W}^{\star}=(w_{2}^{\star},w_{3}^{\star},...,w_{J+1}^{\star})'\)
solve: \begin{equation}
\mathbf{W}=\text{arg}\; \min_{\mathbf{w}:w_{j}\in[0,1],\sum_{j=2}^{J+1} w_{j}=1}\vert\vert\mathbf{X}_{0}\mathbf{W}-\mathbf{X}_{1}\vert\vert
\end{equation}

This is subject to the weights being non-negative and summing up to one,
an important assumption in the classic SCM. This assumption can be
relaxed to allow for non-negative weights, in this thesis we will
however keep it. We can therefore simply estimate the average treatment
effect from (3.1) using these weights: \begin{equation}
\hat{\tau}_{1t}=Y_{1t}^{I}-\sum_{j=2}^{J+1} w_{j}^{\star}Y_{jt}\; \; \text{for}\; \; t>T_{0}
\end{equation}

\subsection*{Inference}

The main idea of inference in synthetic control is using permutation
through the use of placebo effects. A synthetic control unit is
constructed for all the untreated countries in the control group, as if
there was a treatment for these countries. If the magnitude of the
effect for the actually treated unit is extreme compared to the placebo
effective, the effect can be regarded as significant. This may however
not always work great, since it might be hard to get a good
pre-treatment fit for all units in the donor pool. A possiblity of
enhancement is a test statistic which measures the ratio of the
post-intervention fit relative to the pre-intervention fit
\(R_{j}(t_{1},t{2})\), which is defined as the root squared mean
prediction error of the synthetic control. From there it is possible to
compute \(r_{j}\), which measures the quality of the fit in the
post-intervention compared to pre-intervention. It is then also possible
to find a p-value for the permuted test: \begin{equation}
p=\frac{1}{J+1}\sum_{j=1}^{J+1}I_{+}(r_{j}-r_{1})
\end{equation} where \(I_{+}(\cdot)\) is an indicator function that
returns one for non-negative arguments and zero otherwise.

\section{Data}

The data analysis has been carried out in R using the tidysynth package.
The data and R scripts can be found in the corresponding
\href{https://github.com/benediktstelter/bachelor_thesis.git}{GitHub repository \ExternalLink}\hspace{-0.1cm}.

\chapter{Results}

\chapter{Discussion}

\chapter{Conclusion}


 
\backmatter
 
\renewcommand\refname{References}
\printbibliography[title=References]

\chapter{Appendix}
Hier steht ein Anhang.



\chapter{Affidavit}
\thispagestyle{empty}

% Dieser Text entspricht den genauen Vorgaben der Richtlinien für Bachelorarbeiten sowie der Prüfungsordnung (\§ 14a) 
% Stand: November 2014
I affirm that this Bachelor thesis was written by myself without any unauthorised third-party support. All used references and resources are clearly indicated. All quotes and citations are properly referenced. This thesis was never presented in the past in the same or similar form to any examination board. 

\noindent I agree that my thesis may be subject to electronic plagiarism check. For this purpose an anonymous copy may be distributed and uploaded to
servers within and outside the University of Mannheim.

\vspace{2\baselineskip}

\noindent German translation:\\
Ich versichere, dass ich die vorliegende Arbeit ohne Hilfe Dritter und ohne Benutzung anderer
als der angegebenen Quellen und Hilfsmittel angefertigt und die den benutzten Quellen
wörtlich oder inhaltlich entnommenen Stellen als solche kenntlich gemacht habe. Diese Arbeit
hat in gleicher oder ähnlicher Form noch keiner Prüfungsbehörde vorgelegen.

\noindent Ich bin damit einverstanden, dass meine Arbeit zum Zwecke eines Plagiatsabgleichs in
elektronischer Form anonymisiert versendet und gespeichert werden kann.

\vspace{4\baselineskip}
\begin{center}
\parbox{.8\textwidth}{Mannheim, 17/03/2023 \hfill Benedikt Stelter}
\end{center}


 
\end{document}
