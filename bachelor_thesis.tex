% !TEX program = lualatex
% !TEX encoding = UTF-8 Unicode
% !TEX spellcheck = de_DE
% 
% Vorlage für Bachelorarbeiten
% 
% Um die Vorlage mit LaTeX zu erstellen sind folgende Programme aufzurufen:
% > lualatex hauptdatei.tex
% > biber hauptdatei
% > lualatex hauptdatei.tex

\documentclass{scrbook}

% !TEX root = hauptdatei.tex
% !TEX encoding = UTF-8 Unicode
% !TEX spellcheck = de_DE
%
% für mehr Informationen zu einzelnen Paketen siehe zum Beispiel:
% http://texdoc.org/pkg/paketname
% http://ctan.org/pkg/paketname


%% Setzen von Dokumentenoptionen:
%% (aquivalent zu \documentclass[<Optionen>]{...}

%%% Layout-Einstellungen
	\KOMAoptions{ 				% aquivalent zu \documentclass[<Optionen>]{...}
		fontsize=12pt,			% Standartschriftgröße
		bibliography=totoc,	% Bibliografie soll im Inhaltsverzeichnis auftauchen
		headings=normal,		% Größe und Abstand von Überschriften
		toc=listof,				% Verzeichnisse der Gleitumgebungen ins Inhaltsverz.
		toc=indent,				% Inhaltsverzeichnis in hierarchischer Form
		listof=indent,			% andere Verzeichnisse in hierarchischer Form
		listof=totoc,           % andere verzeichnisse im Inhaltsverzeichnis führen
		twoside=false				% enseitiges Layout
	}
	\setcounter{tocdepth}{1} % Ebenentiefe des Inhaltsverzeichnis
	\usepackage{geometry, setspace}
	\geometry{
		paper=a4paper,		% DIN A4 Papier
		hmargin=30mm,			% horizontale Seitenränder
		top=15mm,				% oberer Rand
		bottom=20mm,			% unterer Rand
		includeheadfoot,	% Kopf- und Fußzeilen gehören nicht zum Rand
	}
	\onehalfspacing 		% anderthalbfacher Zeilenabstand
	% Kapitelüberschrift etwas nach oben versetzen:
	\renewcommand*{\chapterheadstartvskip}{\vspace*{-1.48\topskip}}


%% Einstellung der Schriftart:
	\usepackage{lmodern}
	% Alternativ können mit fontspec beliebige im Betriebssystem installierte Schriften verwendet werden:
	%\usepackage{fontspec}
	%\setmainfont{Constantia}
	%\setsansfont{Corbel}
	%\setmonofont{Consolas}
	% Für Serifen in den Überschriften:
	%\addtokomafont{sectioning}{\rmfamily}


%% Einstelen von Kopf- und Fußzeilen:
	\usepackage[headsepline=false]{scrlayer-scrpage}
	% automatisches Füllen der Kopfzeile mit aktuellem Kapitel/Abschnitt:
	\automark[chapter]{chapter}
	% links, mitte, rechts
	\lohead{}
	\cohead{}
	\rohead{}
	% Seitenzahl nur auf plain-Seiten im Fuß
	\lofoot{}
	\cofoot*{\pagemark}
	\rofoot{}
	% Aktivieren des festgelegten Kopfzeile:
	\pagestyle{scrheadings}


%% Sprachauswahl:
	\usepackage{polyglossia}
	\setmainlanguage{english}
	% die Sprache kann im Dokument mit
	% \begin{english} ... \end{english}
	% umgestellt werden


%% Einstellungen zu Zitaten und Bibliografie:
\usepackage{csquotes}

\usepackage[
backend=biber,
bibstyle=apa,
citestyle=apa,
]{biblatex}


\setlength{\bibhang}{15pt}
\defbibenvironment{bibliography}
  {\list
     {}
     {\setlength{\leftmargin}{\bibhang}%
      \setlength{\itemindent}{-\leftmargin}%
      \setlength{\itemsep}{\bibitemsep}%
      \setlength{\parsep}{\bibparsep}}}
  {\endlist}
  {\item}
  \addspace

\setlength\bibitemsep{1.3\itemsep}


	%% sonstige Pakete:
	\usepackage{
		array,		% erweiterte Option für Tabellen
		booktabs,	% schöne Tabellen
		float,		% Platzierung von Gleitobjekten (Abb., Tab.), Eigene Gleitobjekte
		graphicx,	% ermöglicht einbinden von Grafiken mit \includegraphics
		hologo,		% für TeX-Logos
		mathtools,	% Verbesserungen für den Mathesatz (läd u.a. amsmath)
		microtype,	% mikrotypografische Verbesserungen (z.B. optischer Randausgleich)
		paralist,	% platzsparende Listen mit compactitem
		xcolor,		% Verwendung von Farbe
	}


%% LaTeX sucht nach Bildern an den hier angegebenen Stellen:
	\graphicspath{{./images/},{./}}


%% automatische PDF-Verlinkungen im Dokument:
	\usepackage[
		colorlinks=false,		% Links nicht farbig hervorheben
		pdfborder={0 0 0},		% links nicht durch PDF-Kasten hervorheben
	]{hyperref}






\addbibresource{thesis.bib}


\usepackage{tikz}

\usepackage{array,makecell}
\usepackage{caption}


\usepackage{tabularx}
    \renewcommand\tabularxcolumn[1]{m{#1}}
    \newcolumntype{C}{>{\centering\arraybackslash}X}

\newcommand{\ExternalLink}{%
    \tikz[x=1.2ex, y=1.2ex, baseline=-0.05ex]{% 
        \begin{scope}[x=1ex, y=1ex]
            \clip (-0.1,-0.1) 
                --++ (-0, 1.2) 
                --++ (0.6, 0) 
                --++ (0, -0.6) 
                --++ (0.6, 0) 
                --++ (0, -1);
            \path[draw, 
                line width = 0.5, 
                rounded corners=0.5] 
                (0,0) rectangle (1,1);
        \end{scope}
        \path[draw, line width = 0.5] (0.5, 0.5) 
            -- (1, 1);
        \path[draw, line width = 0.5] (0.6, 1) 
            -- (1, 1) -- (1, 0.6);
        }
    }


\usepackage{booktabs, caption, makecell}
\renewcommand\theadfont{\bfseries}
\usepackage{threeparttable}

\counterwithout{footnote}{chapter}


%% Alle wichtigen Einstellungen sind in der Datei einstellungen.tex getätigt
%% und können dort verändert werden.

\begin{document}
	
\frontmatter
\begin{titlepage}

\begin{center}

\vspace*{1,2cm}

\huge {\bfseries The impact of vaccine lotteries on COVID-19 vaccination
rates: Evidence from Poland}\\[1.8cm]

\Large {Bachelor Thesis}\\[1cm]

\large {Department of Economics}\\[0.2cm]

\large {University of Mannheim}\\[0.5cm]

\end{center}

\vfill

\noindent submitted to:\\
Prof.~Achim Wambach, PhD / Sabrina Schubert\\[1cm]
submitted by:\\
Benedikt Stelter\\[1cm]
Student ID: 1731015\\
Degree Programme: Bachelor of Science in Economics (B.Sc.)\\[1cm]
Address: Meerfeldstr. 11, 68163 Mannheim\\
Phone: +49 176 95741248\\
E-Mail: benedikt.stelter@students.uni-mannheim.de\\[1cm]
Mannheim, 17/03/2023

\setcounter{page}{0}

\end{titlepage}

  \tableofcontents


%% Abbildungsverzeichnis
\listoffigures

%% Tabellenverzeichnis
\listoftables


\mainmatter

\chapter{Introduction}

In March 2020, the world came to a sudden stop due to the spread of
COVID-19. Governments took drastic and unprecedented steps to slow the
spread of the virus: Shops were shut down, schools and universities were
closed and millions of workers had to work from their home office or
kitchen desk. Many of these measures to counteract COVID-19 had negative
side effects: They destroyed billions of dollars of wealth, led to
unemployment and strangled the mental health of many. Most importantly
however, apart from all measures to stop its spread, the COVID-19
pandemic itself cost the lives of millions of people around the world.

Just a year after the first COVID-19 cases appeared in China, the first
vaccines against COVID-19 were approved. With them came the hope of a
return to normality and a severe reduction of the death toll of the
pandemic. At the beginning of the worldwide vaccination effort, the
producers of the vaccines were clearly not able to meet the world demand
at the necessary speed and scale. Therefore, a shortage of vaccines lead
to rationing: Vaccines were only provided to the most vulnerable groups
of the population, such as healthcare workers or the elderly. When
vaccines became widely available, policymakers however had to learn that
a considerable share of people were hesitant or unwilling to receive the
shot. Globally, the COVID-19 vaccine was one of the most relevant and
hotly contested topics of 2021, demonstrated by Google
Trends\footnote{see \url{https://trends.google.de/trends/yis/2021/GLOBAL/}},
which places the COVID vaccine as the 3rd most searched news story of
the entire year.

The problem of vaccine hesitancy led governments to think in
new\footnote{As we will see, the idea of incentives and lotteries in public health is not entirely new.}
ways again. From beer to bratwurst and fishing license to lotteries:
Many different incentives were introduced on national and local levels,
in order to persuade people of the COVID-19 vaccine. Lotteries might be
especially interesting, since they include an element of uncertainty. A
vaccine lottery, in the context of this thesis, refers to a lottery with
cash or non-cash prizes, in which vaccinated people could participate at
no additional cost, thereby acting as a possible reward for vaccination.
Policymakers hoped that this would persuade additional people to get
vaccinated, increasing the total vaccination rate of the population.

This thesis will investigate the impact of such lotteries. It therefore
tries to answer the question: What are the effects of vaccine lotteries
on the share of the population vaccinated against COVID-19? The
empirical analysis using a vaccine lottery in Poland finds no evidence
for a significant increase in the share of the population fully
vaccinated against COVID-19. It however suggests that it might have
incentivized some people who would have gotten vaccinated anyway to do
this earlier, to take advantage of the lottery, thereby temporarily
increasing the vaccination rate by around 6 percentage points.

As we will see in section 2.1, there is already a large literature on
COVID-19 vaccine lotteries, especially in North America. So far, there
has however been no study evaluating the Polish vaccine lottery in
detail. This thesis therefore adds a new case study to the existing
literature.

In order to to spur the slow-moving national vaccination campaign, the
Polish government announced the lottery on May 25,
2021\footnote{This was around five months after the authorisation of the first COVID-19 vaccines in the EU and the subsequent start of the Polish vaccination campaign.}.
It was open from July 1, 2021 until September 30, 2021. The main prize
of the lottery was a cash prize of one million PLN, but it also included
smaller cash and non-cash prizes and a lottery-like incentive scheme for
municipalities. The empirical analysis was conducted using the synthetic
control method based on \textcite{abadie_economic_2003}, a popular
method for comparative case studies. Out of a donor pool of 6 other
Eastern European countries and using additional predictor variables, a
synthetic control unit (``Synthetic Poland'') was constructed to
evaluate the (causal) effect of the policy. Daily data on vaccination
rates, published by the respective governments/government agencies and
prepared by Our World in Data and additional data from Eurostat and the
Wellcome Global Monitor were exploited for the analysis.

\chapter{Background}

\section{Literature review}

\subsection*{Incentives in public health}

Governments are facing public health challenges at every corner. But how
can they motivate people to live healthier lives? Behavioural economics
can offer several ways to influence the decisions of individuals,
probably the least severe form are nudges.

Nudging is a concept mainly brought to the public through the work of
\textcite{thaler_nudge_2008}. They define a nudge as an intervention
that ``alters people's behavior in a predictable way without forbidding
any options or significantly changing their economic incentives''
(p.~6). Nudging can be applied in various ways, it is however often
connected to lifestyle topics, such as nutrition and diet. For example,
in an experiment in Denmark \parencite{friis_comparison_2017}, the aim
was to promote the consumption of vegetables in the setting of a
self-serving buffet, which included salad and other dishes in large
bowls. As a nudge, the food environment was changed by arranging green
plants and herbs around the food bowls. In a second experiment, salad
was pre-portioned into smaller take-away bowls. The results showed that
the intake of energy from vegetables of the participants can be
increased by pre-portioning the salad. Although many studies evaluating
the impact of nudges find positive effects of the respective
interventions, these results should be interpreted with caution, since a
lot of studies, including \textcite{friis_comparison_2017}, were
conducted in the lab and may therefore not be reproducible in a
real-world setting \parencite{ledderer_nudging_2020}.

Besides nudging, policymakers could try to change economic incentives.
Governments have been using taxation as a means of influencing the
behaviour of individuals for a long time, for example in alcohol and
tobacco policy. Both in developed and developing country, it has been
shown that raising prices leads to reduced consumption of tobacco
(\cite{yeh_effects_2017}; \cite{immurana_effects_2021}).

Beyond taxation, policymakers could also use other financial incentives
to motivate changes of individual behaviour to tackle public health
challenges, such as (small) cash payments or lotteries. There have been
many studies evaluating such possible schemes, typically using
randomised control trials. Several meta-analyses found that such
incentives can be successful in inducing behaviour changes. For example,
\textcite{giles_effectiveness_2014} evaluated 16 studies on issues such
as smoking cessation, health screenings, physical activity and
vaccinations. The authors found that financial incentives are more
effective than no intervention at encouraging healthy behaviour. This
finding is also confirmed by \textcite{mantzari_personal_2015}, who
evaluated 34 studies and additionally concluded that this effect is
stronger for the most deprived individuals, thereby possibly reducing
health inequalities.

There is also some literature on lotteries as an incentive in public
health, apart from COVID-19 vaccine policy.
\textcite{bjorkman_nyqvist_incentivizing_2018} find that the
introduction of a lottery program reduced HIV incidence in Lesotho.
Lotteries can also successfully increase cycling and walking activity
(\cite{ciccone_using_2021}; \cite{patel_randomized_2018}) and
participation in chlamydia screenings \parencite{niza_vouchers_2014}.

\subsection*{Incentives to increase COVID-19 vaccination rates}

In march 2020, when the COVID-19 pandemic struck the world, it was not
foreseeable that vaccines would be available in just about nine months.
When they were available and hailed as the possible end of the pandemic,
it may not have been clear that rolling out vaccines was so difficult.
In countries around the world, there was a considerable amount of
vaccine hesitancy. Dealing with this hesitancy has led policymakers and
researchers to thinking about how to increase vaccination rates, by
means of vaccination mandates and passports, nudges, cash and non-cash
prizes and lotteries.

A very widely used tool, especially in Europe and the US, were
vaccination passports. The access to social gatherings/places
(e.g.~restaurants, bars, clubs, stadiums), international travel and
quarantine regulations were subject to vaccination against COVID-19.
Besides ethical concerns, for example potential disqualification of
minorities from social life because of a historically higher distrust in
government \parencite{gostin_digital_2021}, the evidence on the
effectiveness of such regulations is mixed. A synthetic control analysis
of six countries found that COVID passports were successful in
increasing daily vaccinations in countries with lower than average
vaccination rates. In more average countries (Germany and Denmark), such
regulations were less successful \parencite{mills_effect_2022}. These
cross-country comparisons should however be treated with caution, since
there are large variations across countries when it comes to the extent
of the use of COVID passports. While a comparison of Poland (very
restricted use of COVID passport) and Lithuania (wide use of COVID
passport) suggests a positive effect of passports on vaccination rates
\parencite{walkowiak_covid-19_2021}, it has to be noted that COVID
passport may also have negative effects on vaccine uptake, since
frustration about reduced autonomy might lower willingness to get
vaccinated \parencite{porat_vaccine_2021}.

Besides the carrot-and-stick approach, governments have also used
smaller nudges and financial incentives, such as cash payments and
non-cash rewards to increase COVID-19 vaccination rates. Probably the
most influential study to date on cash payments and nudges was carried
out in Sweden \parencite{campos-mercade_monetary_2021}. Using a
randomised control trial, the authors find that even ``small'' cash
payments of around 24 US dollars can significantly increase vaccination
rates, while small nudges could not increase vaccination rates. While
some studies also suggest a measurable positive effect of cash payments
on vaccination rates (\cite{wong_guaranteed_2022};
\cite{kluver_incentives_2021}; \cite{kim_vaccination_2021-1}), there is
also evidence against this effect. In another study, the results
indicated that neither behavioural nudges (text messages) nor cash
payments could increase vaccination rates among the hesitant citizens
\parencite{jacobson_can_2022}. \textcite{sprengholz_money_2021} also
find that cash incentives do not increase the willingness to be
vaccinated.

There is also a vast literature specifically on COVID-19 vaccine
lotteries, also evaluating real-world lotteries.
\textcite{dube_exploring_2022} analyse the effectiveness of a vaccine
lottery in Québec (Canada) and find a relatively small impact on
vaccination rates. A survey in Australia found that the vaccine lottery
there was successful in increasing willingness to be vaccinated
\parencite{jun_association_2022}. The majority of studies on COVID-19
vaccine lotteries however dealt with lotteries in US states. Studies on
vaccine lotteries in Louisiana and Massachusetts found different
effects. Whereas the lottery in Louisiana increased vaccine uptake
\parencite{wang_moving_2023}, a vaccine lottery in Massachusetts did not
significantly increase vaccination rates although prizes were higher
\parencite{kim_did_2023}. There are also several papers investigating
multiple state COVID-19 lotteries at once. All of these papers find that
most, but not all of the lottery programs were successful in increasing
vaccine uptake (\cite{robertson_are_2021};
\cite{acharya_implementation_2021}; \cite{fuller_assessing_2022}).

A specific focus can be observed with respect to Ohio. There is a quite
large number of studies evaluating the ``Ohio Vax-A-Million'' lottery,
which was the first COVID-19 vaccine lottery in the US. In total, a
majority of the reviewed literature casts a positive light on the
efficacy of the lottery. While \textcite{mallow_covid-19_2022} find a
positive effect of the lottery on vaccination rates, another study
cannot support this view \parencite{walkey_lottery-based_2021}. There
have also been four studies evaluating the lottery using the SCM. These
studies construct a synthetic Ohio out of a donor pool of other US
states. Three of these studies find small positive effects of the
lottery of vaccine uptake (\cite{brehm_ohio_2022};
\cite{barber_conditional_2022}; \cite{sehgal_impact_2021}) and one does
not find a robust effect \parencite{lang_did_2022}.

So far, there has been one study dealing with the vaccine lottery
analysed in this thesis. \textcite{kuznetsova_effectiveness_2022}
shortly evaluate different vaccine incentives across Europe, including
the Polish policy. By applying an interrupted time series analysis based
on an ARIMA approach, the authors suggest a slight positive effect of
this lottery. There has however not been a study dealing in detail with
the Polish lottery yet.

\section{Institutional background}

The COVID-19 policies in Poland (especially with respect to vaccines)
can be considered as relatively ``relaxed''. Poland did not widely use
the EU COVID passport with respect to social life (citation here).
Instead, it was only used for travelling/entry into the country. The
Polish vaccination campaign started on December 27, 2020 with the
vaccination of a nurse (citation for total vaccination campaign). Until
September 2021 when two vaccines were no longer used, all of the
vaccines approved by the EU (AstraZeneca, Pfizer/BioNTech, Moderna and
Johnson \& Johnson) were used for vaccinations. Citizens who wanted to
get vaccinated had to register and select their preferred vaccine prior
to the vaccination. In May 2021, the waiting time between the first and
the second dose (Johnson \& Johnson required only one dose) was set at
five weeks. In the first weeks and months of the vaccination campaign,
vaccines were only available for health care workers and senior
citizens. By the start of May 2021 however, vaccines were widely
available (cite here), but not enough citizens were willing to get
jabbed. In a relatively short time, Poland therefore moved from a
problem of scarcity to the problem of vaccination hesitancy.

In general, vaccination rates in Poland have been low compared to some
western European countries, including neighbour Germany. In the context
of eastern Europe, vaccination rates have however not been exceptionally
low. Many eastern European countries experienced relatively low
vaccination rates and a hesitancy to get vaccinated within their
respective population. In the media, a general distrust into the
government and a lack of educational campaigns
\parencite{noauthor_polands_2021} and a chaotic and conflicting
communication by the government \parencite{wanat_polands_2021} have been
cited as potential reasons of the low vaccine uptake in Poland. Many
other eastern European countries also refrained from a large-scale use
of COVID passports and the carrot-and-stick approach in general.
Therefore, to increase vaccination rates, the only remaining options are
campaigns/nudges and providing incentives. This has actually been
implemented in several ways, e.g.~Slovakia and Bulgaria also adopted
lotteries at some point and Lithuania (?) provided cash incentives.

\renewcommand*{\arraystretch}{1.5}
\begin{table}[! htbp]\centering \caption{Prizes of Polish vaccine lottery}
\label{table:summarystat}
\begin{threeparttable}
\begin{tabularx}{10.5cm}{c|c|c}
\toprule\midrule
 & \thead{Cash prizes} & \thead{Non-cash prizes}\\ \midrule
Instant & \(13,000*500\) PLN & - \\
 & \(39,000*200\) PLN & \\ \hline
Weekly & \(60*50,000\) PLN & 720 electric scooters \\  \hline
Monthly & \(6*100,000\) PLN & 6 small vehicles \\ \hline
Main & \(2*1,000,000\) PLN & 6 middle class vehicles \\
\bottomrule
\end{tabularx}
\begin{tablenotes}
      \item \footnotesize Source: Service of the Republic of Poland
    \end{tablenotes}\end{threeparttable}
\label{table2}
\end{table}

\renewcommand*{\arraystretch}{1}

The empirical analysis will be based on a vaccine lottery implemented in
Poland from July 1, 2021 to September 30, 2021. It was announced on May
25 , 2021 \parencite{charlish_poland_2021}. The policy had two main
elements: A lottery for all adult fully vaccinated people (two doses) in
Poland \parencite{service_of_the_republic_of_poland_national_2021} and a
lottery-like monetary incentive scheme for municipalities
\parencite{service_of_the_republic_of_poland_competitions_2021}. The
main price of the lottery was a cash prize of 2 * one million
zloty\footnote{At the time of announcement on 25/05/2021, this was equal to around 220,000€, with an exchange rate of around 0.22 PLN to EUR},
but it also included smaller monthly, weekly and daily cash prizes and
non-cash prizes (cars and electric scooters) with a total volume of 140
million zloty, as depicted in table 2.1. Citizens were able to enter the
lottery both online and by phone. It was organized by the state-owned
polish lottery company \textit{Totalizator Sportowy}, which also
operates other popular lotteries in Poland.

Poland's lottery can be seen as a mixture of different brands of
lotteries. Especially in the US, state governments have focused on
lotteries with high rewards and relatively low winning probabilities
(e.g.~Ohio, Massachusetts with prizes of one million USD). A completely
different concept would be the use of relatively low prizes (e.g.~below
1000 USD) with relatively high winning probabilities (citation/example
here). The Polish policy included both relatively small prizes (instant
prizes) but also quite large prizes (main/final draw), thereby combining
the best of both worlds.

As part of the monetary incentive scheme for municipalities, the
municipality with the highest percentage of the vaccinated in the
country received two million zloty. Three other municipalities who had
the highest percentage of the vaccinated in their comparison
group\footnote{There were three groups: Municipalities with a population of up to 30,000, cities with a population of 30,000 - 100,000 and large cities with a population above 100,000.}
received one million zloty each. The 500 quickest other municipalities
reaching a vaccination rate of 67\% won 100.000 zloty each.

One obvious question with respect to the synthetic control is the time
of intervention. Although the actual lottery started on July 1, we will
choose the time of announcement as the point of intervention, since the
time of being fully vaccinated is not relevant for the entry into the
lottery. If the lottery has an effect, we will expect it to be
observable from the time of announcement. On the other hand, if we were
to set the intervention time at July 1, we might have the problem of a
so-called anticipation effect, impacting the credibility of the
synthetic control negatively. This will be further discussed in chapter
3.1. As a robustness check for the result we obtain, we will however
also estimate a synthetic control using July 1 as the time of
intervention.

\chapter{Method and data}

\section{Synthetic control method}

The synthetic control method is a relatively new method in causal
inference. It was first established by \textcite{abadie_economic_2003}
in order to investigate the economic effects of terrorism in the basque
country and further developed by \textcite{abadie_synthetic_2010}.
\textcite{abadie_using_2021} is a summary of these prior developments as
well as the requirements and inference methods and can be regarded as
the basis of this section.

Synthetic control methodology has been applied widely in economics, but
also in other (social) sciences such as political science. Its specific
case of application are comparative case studies, where a specific case,
e.g.~a policy or intervention of any form are compared to another case,
for example without a similar policy (?). One idea in the context of
comparative case studies is difference-in-difference. It might be
straightforward to use matching to find the closest/most similar unit to
the treated unit and take this as a control, allowing to estimate the
average treatment effect of the policy by simply taking the difference
(citation for matching). Finding a single unit which matches well might
however in many cases be difficult, especially when making cross-country
comparisons. The synthetic control method comes into play here and
extends the difference-in-difference method. The very basic idea is to
create a synthetic control unit as a combination of multiple control
countries, to then estimate the average treatment effect (ATE) of the
policy/intervention. For instance, it has been used to evaluate the
effects of European integration \parencite{campos_institutional_2019} or
the effect of natural disasters on economic growth
\parencite{cavallo_catastrophic_2013}, by constructing synthetic control
countries without European integration or natural disasters.

\subsection*{Requirements}

There are some important requirements that need to be fulfilled, in
order to obtain a valid and interpretable synthetic control. These have
been outlined in detail by \textcite{abadie_using_2021} along with the
formal definition and the inference. The first one is that the evaluated
policy/intervention has a sufficiently large effect. When the effect of
an intervention is too small, it may not be possible to distinguish this
effect from other shocks to the outcome variable. Additionally, the
volatility of the outcome variable should not be too high, to prevent
over-fitting.

The second one is that there exists a suitable comparison/control group.
Countries that are also subject so similar interventions or other shocks
to the outcome variable in the given time frame should be excluded from
the donor pool. What this means for the analysis in this thesis is
outlined in section 3.1. Furthermore, we should try to select countries
that are not too different from the treated country for the donor pool,
to prevent interpolation bias(cite or write about interpolation bias).

Another important requirement is that there is no anticipation effect,
so that citizens do not anticipate the enactment of a policy. If there
were an anticipation, we would expect our synthetic control to become
biased and not useful for estimating the causal effect of the treatment.
This should however not be expected as a problem in the Polish lottery
application, since there is no possible reason for anticipation (there
were no prior lotteries in Europe and no rumors about the lottery before
the announcement (check again)). This will be picked up again in chapter
4, as a robustness check of the obtained synthetic control.

Next, it is crucial that we do not have any spillover effects on
untreated units. If this were the case, the lottery in Poland would have
effects on vaccination rates of donor pool countries. This requirement
should however be fulfilled in this application, there is at least no
possible theoretical argument for such a relationship.

In terms of data, it is important that we have enough pre-intervention
values of the outcome variable, so that the synthetic control can
approximate these values well. A larger number of pre-intervention
outcomes reduces the bias of the synthetic control, making it more valid
(sounds weird).

The fulfillment of these assumptions/requirements will be further
discussed in section 5.

\subsection*{Formal definition}

What we observe is some outcome variable \(Y_{jt}\) for \(J + 1\) units
from \(t=1\) to \(T\). The first unit (\(j = 1\)) is the treated unit
for \(t>T_{0}\) while all other units are untreated units. The
intervention (in this analysis the announcement of the lottery) occurs
at \(T_{0}+1\), meaning that there are \(T_{0}\) pre-intervention time
periods. We also observe \(k\) predictors, which include the
pre-intervention observations of the outcome variable
(\(Y_{j1},Y_{j2},...,Y_{jT_{0}}\)) and additional time invariant unit
level characteristics \(\mathbf{Z}_{j}\) (covariates, as outlined in
section 3.2). These \(k\) total predictors can be summarised by the
vectors
\(\mathbf{X}_{j}=(Y_{j1},Y_{j2},...,Y_{jT_{0}},\mathbf{Z}_{j}')'\) for
units \(j=1,...,J + 1\). It therefore follows that the \((k\times J)\)
matrix
\(\mathbf{X}_{0}=(\mathbf{X}_{2},\mathbf{X}_{3},...,\mathbf{X}_{J + 1})\)
captures all the predictors of the untreated
units\footnote{\(\mathbf{X}_0=
\begin{bmatrix}
Y_{21} & Y_{31} & \dots & Y_{J+11}\\
Y_{22} & Y_{32} & \dots & Y_{J+12}\\
\vdots & \vdots & \ddots & \vdots\\
Y_{2T_{0}} & Y_{3T_{0}} & \dots & Y_{J+1T_{0}}\\
\mathbf{Z}_{2}' & \mathbf{Z}_{3}' & \dots & \mathbf{Z}_{J + 1}'
\end{bmatrix}\)}.

The average treatment effect (causal effect) is defined as the
difference between the potential outcome of the treated unit with
intervention, which can be defined as \(Y_{1t}^{I}\), and its potential
outcome without the intervention, \(Y_{1t}^{N}\): \begin{equation}
\tau_{1t}=Y_{1t}^{I}-Y_{1t}^{N}\; \; \text{for}\; \; t>T_{0}
\end{equation} By definition, the outcome with intervention is known and
the outcome without intervention is hypothetical for \(t>T_{0}\) for the
treated unit. Therefore, to estimate the causal effect, it is sufficient
to estimate \(Y_{1t}^{N}\). How can this problem be solved?

The first straightforward idea might be to use a
difference-and-difference with matching approach and choose the closest
unit \(j^{*}\), the so-called best single control
\parencite{doudchenko_balancing_2016}, which solves: \begin{equation}
j^{*}=\text{arg}\; \min_{j>1}\vert\vert\mathbf{X}_{j}-\mathbf{X}_{1}\vert\vert
\end{equation} Taking a difference then results in an estimator for the
average treatment effect: \begin{equation}
\hat{\tau}_{1t}=Y_{1t}^{I}-Y_{j^{*}t}
\end{equation} Using a single control approach however does not seem
like a desirable estimation technique. Firstly, in many cases (including
the Polish lottery) a single control does not achieve a good
pre-treatment fit, especially when cross-unit differences are relatively
large. Secondly, a difference in difference approach requires the
parallel trend assumption, which is very hard to justify. Parallel trend
requires that in the absence of treatment, the difference between
treatment and control group is constant over time (possibly cite here).
In the application of the Polish vaccine lottery, this would mean that
the difference in the vaccination rates of Poland and country \(j^{*}\),
the closest unit pre-intervention, would be constant if there were no
lottery. This does not seem realistic, since there are many different
factors influencing vaccination rates in each country, thereby making
changes in the difference relatively likely.

The synthetic control method developed by
\textcite{abadie_economic_2003} proposes to use a weighted average of
donor pool units as a synthetic control, thereby not experiencing the
problems of a single control estimation (Synthetic control does not
require parallel trend assumption). The synthetic control and the
following estimator for the average treatment effect are therefore
defined as: \begin{equation}
\hat{Y}_{1t}^{N}=\sum_{j=2}^{J+1} w_{j}Y_{jt}
\end{equation} \begin{equation}
\hat{\tau}_{1t}=Y_{1t}^{I}-\hat{Y}_{1t}^{N}
\end{equation} The weights \(\mathbf{W}=(w_{2},w_{3},...,w_{J+1})'\) are
chosen, such that the synthetic control matches as closely as possible
the pre-intervention path of the predictors of the outcome variable for
the treated unit (possibly cite). Therefore the weights have to be
chosen so that they minimize this difference. The optimal weights
\(\mathbf{W}^{\star}=(w_{2}^{\star},w_{3}^{\star},...,w_{J+1}^{\star})'\)
solve: \begin{equation}
\mathbf{W}^{*}=\text{arg}\; \min_{\mathbf{w}:w_{j}\in[0,1],\sum_{j=2}^{J+1} w_{j}=1}\vert\vert\mathbf{X}_{0}\mathbf{W}-\mathbf{X}_{1}\vert\vert
\end{equation}

(Potentially include factor model) This is subject to the weights being
non-negative and summing up to one, an important assumption in the
classic SCM. This assumption can be relaxed to allow for non-negative
weights, in this thesis we will however keep it. We can therefore simply
plug in these weights to obtain the estimated average treatment effect
from (3.5): \begin{equation}
\hat{\tau}_{1t}=Y_{1t}^{I}-\sum_{j=2}^{J+1} w_{j}^{\star}Y_{jt}\; \; \text{for}\; \; t>T_{0}
\end{equation}

\subsection*{Inference}

Based on \textcite{abadie_synthetic_2010}, the most common way of
inference in synthetic control is using permutation through the use of
placebo effects. A synthetic control unit is constructed for all the
untreated countries in the control group, as if there was a treatment
for these countries. If the magnitude of the effect for the actually
treated unit is extreme compared to the placebo effects, the effect can
be regarded as significant. In order to do this analysis, the gaps
between the synthetic control and the actual outcome can be plotted for
all selected countries, in order to visually compare the size of the
effects. One possible problem of this concept might be that it could
possibly be difficult to obtain a good pre-treatment fit for all units
in the donor pool, especially with a relatively small donor pool.
Additionally, this way of constructing inference is in some sense
``blurry'', since it does not rely on quantitative measure.

A possibility of quantification is a test statistic which measures the
ratio of the post-intervention fit relative to the pre-intervention fit
(possibly cite 2010 abadie here). The root mean squared prediction error
of the synthetic control is then defined as: \begin{equation}
R_{j}(t_{1},t_{2}) =(\frac{1}{t_{2}-t_{1}+1}\sum_{t=t_{1}}^{t_{2}}(Y_{jt}-\hat{Y}_{jt}^{N})^2)^\frac{1}{2}
\end{equation} From there it is possible to compute \(r_{j}\), which
measures the quality of the fit in the post-intervention period compared
to pre-intervention and is given by the ratio of the post-intervention
RMSPE and pre-intervention RMSPE: \begin{equation}
r_{j}=\frac{R_{j}(T_{0}+1,T)}{R_{j}(1,T_{0})}
\end{equation} It is then possible to compute a p-value for the permuted
test: \begin{equation}
p=\frac{1}{J+1}\sum_{j=1}^{J+1}I_{+}(r_{j}-r_{1})
\end{equation} where \(I_{+}(\cdot)\) is an indicator function that
returns one for non-negative arguments and zero otherwise.

\subsection*{Extensions}

One possible extension, proposed by
\textcite{doudchenko_balancing_2016}, is to allow for the weights of the
synthetic control to be negative.

Several extensions to to the classic SCM (outlined so far) have been
proposed in the last years. \textcite{ben-michael_augmented_2021} have
proposed an augmented synthetic control based on ridge regression, a
regression technique which produces better results under
multicollinearity (when predictor variables are correlated with each
other). The augmented synthetic control is aimed at applications in
which a sufficiently good pre-treatment fit cannot be achieved and a
reduction of the following bias is desired.

\section{Data}

First, we will estimate an RDD model to estimate the effect of the
lottery on COVID-19 cases in Poland, similar to the analysis by
\textcite{kuznetsova_effectiveness_2022}. For this, time series data on
COVID-19 cases by ``Our world in data'' will be utilized.

In the main synthetic control analysis, the impact of the policy on the
vaccination rates will be investigated. Data for the outcome variable
(\(Y_{jt}\)) vaccination rate (both one vaccination and fully
vaccinated) is taken from a data set created by Our World in Data
\parencite{mathieu_global_2021}. This very large data set is a
collection of vaccination rates and other vaccination statistics from
all countries of the world, coming directly from the respective
government/government agency and summarized by Our World in Data. If the
data is provided by the governments, it can give us a daily time-series
of vaccination rates for every country of the world.

\begin{figure}[h]
\caption{Share of population fully vaccinated in Poland}

\begin{center}\includegraphics{bachelor_thesis_files/figure-latex/unnamed-chunk-2-1} \end{center}
\end{figure}

Figure 3.1 shows the resulting graph of the share of fully vaccinated in
Poland. It can be seen that there was a period of relatively high daily
vaccinations in which vaccination rates increased drastically. Before
and after this period, the growth in the vaccination rate was however
relatively slow.

One of the most important aspects in the application of SCM is the
choice of the donor pool. As outlined in 3.1, the treated country should
not be an outlier compared to the control countries. It is therefore
sensible to select countries that are similar to Poland, both in general
and with respect to vaccination rates. Therefore, twelve central/eastern
European countries with sufficient data availability (AT, BG, CZ, EE,
GR, LV, LT, HR, HU, RO, SI, SK) are initial candidates for the donor
pool. From this list, several countries are dropped who experienced
similar interventions or shocks to the vaccination rate in the given
time frame. Greece (cash incentive of 150€ for young people), Czechia
(holiday incentive), Austria (need to find a reason here), Lithuania
(cash incentive + other incentives), Romania (cash and lottery
incentive). Estonia, Czechia and Slovenia implemented some small
incentive schemes for
doctors\footnote{Doctors were offered a cash incentive for a specific number of vaccinations}
and state employees\footnote{Additional holiday}, these are however not
considered as a big enough shock to vaccination rates and Estonia,
Czechia and Slovenia will not be dropped. Croatia and Hungary are
removed because of too many missing values, since a sensible trend can
not be constructed using interpolation. Slovakia (nochmal checken) only
reported weekly values, a reasonable interpolation was however possible.
The donor pool then consists of six countries: Bulgaria, Czechia,
Estonia, Latvia, Slovakia and Slovenia.

There are however missing values for several countries. Sometimes these
missing values follow a specific pattern (e.g.~Poland: values are
missing on Sundays) while there is no specific pattern for some other
countries. Linear interpolation is used to replace the missing values,
by drawing a straight line between the two adjacent data points
(citation here). Other imputation techniques such as spline
interpolation\footnote{Spline interpolation estimates missing values such that the curvature of the time series is minimized (need citation here).}
were also considered. A last observation carried forward (LOCF) or a
mean imputation do not make any sense in the setting of a vaccination
rate, which is an increasing function. The differences between linear
and spline interpolation are relatively minor in this specific
application.

\begin{table}[! htbp]\centering \caption{Predictors of selected countries}
\label{table:summarystat}
\begin{threeparttable}
\begin{tabular}{l c c c}
\toprule\midrule
 & \thead{Poland}
 & \thead{Synthetic Poland} & \thead{Mean donor}\\ \midrule
GDP per capita\tnote{a} & $12,810$ & $15,642.287$ & $14,428.333$ \\ 
Influenza vaccination rate\tnote{b} & $0.104$ & $0.133$ & $0.139$ \\ 
Population density\tnote{c} & $123.600$ & $87.871$ & $79.667$ \\ 
Share with tertiary education\tnote{d} & $0.289$ & $0.283$ & $0.286$ \\
Share of elderly\tnote{d} & $0.182$ & $0.185$ & $0.198$ \\ 
Trust in science\tnote{e} & $0.872$ & $0.895$ & $0.891$ \\ 
\bottomrule
\end{tabular}
\begin{tablenotes}\footnotesize
\item[a] in USD, 2020 (Eurostat)
\item[b] among elderly (over 64), 2019 (Eurostat)
\item[c] in persons per \(\text{km}^{2}\), 2019 (Eurostat)
\item[d] over 64 years, 2020 (Eurostat)
\item[e] 15 to 64 years, 2020 (Eurostat)
\item[f] 2020 (Wellcome Global Monitor)
\end{tablenotes}
\end{threeparttable}
\label{table2}
\end{table}

In order to find the best fit for the weights, additional predictors
(\(mathbf{Z}_{j}\)) are used. The choice of these variables is
potentially very important in determining the optimal weights of the
synthetic control unit, as seen in section 3.1. GDP per capita, the
share of elderly (over 65), the share of people (15-64) with tertiary
education, population density and the year of entry into the EU all
adjust for general country specific differences, but partly also for
differences in vaccine uptake. Income, age, education and the proximity
to the nearest vaccination center (it is assumed that countries with
higher population density have - on average - a closer proximity to
vaccination centers) are all relevant determinants of COVID-19 vaccine
uptake (\cite{viswanath_individual_2021}; need to check + possibly
additional citation). The share of elderly vaccinated against Influenza
and trust in science are possibly also important indicators of ``vaccine
openness''. We would expect countries with higher Influenza vaccination
rates (before the spread of COVID-19) also to have higher COVID-19
vaccination rates, since the population might generally be more open to
the basic idea of vaccinations and that there might be a more
sophisticated culture of health prevention (too unspecific). A similar
reasoning applies to trust in science. When citizens generally place
more confidence in scientists, they would also be more likely to be open
to the idea of vaccinations.

\begin{table}[! htbp]\centering \caption{Composition of synthetic Poland}
\label{table:weightssynth}
\begin{threeparttable}
\begin{tabular}{l c c c c c c c c c c}
\toprule\midrule
\thead{Country} & & & & & & & & & & \thead{Weight}\\ \midrule
Bulgaria & & & & & & & & & & 0.001 \\ 
Czechia & & & & & & & & & & - \\
Estonia & & & & & & & & & & - \\
Latvia & & & & & & & & & & 0.269 \\ 
Slovenia & & & & & & & & & & 0.248 \\ 
Slovakia & & & & & & & & & & 0.482 \\  
\bottomrule\addlinespace[1ex]
\end{tabular}
\end{threeparttable}
\label{table2}
\end{table}

Ideally, the use of a ``political variable'' would have been a good
idea, since differences in vaccine uptake exist across party
preferences. This problem might be the most well-known in the US
\parencite{ruiz_predictors_2021}, but it is also believed to be a
relevant predictor of vaccine hesitancy (even before COVID-19) in Europe
(\cite{schernhammer_correlates_2022}; \cite{kennedy_populist_2019}).
When using other European countries as a donor pool, it is however very
difficult to compare political beliefs across countries, mainly because
of the large differences in party ideologies across countries. One idea
would be to use the vote share for parties along the groups in the
European Parliament, there are however several problems with that.
Firstly, there are relatively large differences between parties within
certain
groups\footnote{For example, the Renew Europe group in the EP consists of liberal parties in a very broad sense, including both left-wing liberal parties and liberal-conservative parties.}
and secondly, elections and surveys are very volatile in many European
countries compared to e.g.~some US states (if left, need citations).

Another possible predictor which was considered is trust in government.
The data on this is also available from the Wellcome Global Monitor. It
is ultimately however not selected, for several reasons. Firstly, trust
in government is often highly volatile. For example, if a very unpopular
government is replaced by a new government, this might increase the
trust into public sector dramatically in a very short time span. The
problem of a comparably high volatility should be especially pronounced
in a time of crisis like the COVID-19 pandemic, where temporarily very
high/very low infection rates can lead to quick changes in public
opinion. While there also exist deep-rooted cross-country differences in
trust in government (e.g.~possibly higher distrust in former soviet
influenced countries compared to western countries (possibly need
citation here)), the presence of the described volatility means that the
variable trust in government will not be used.

\noindent The data analysis has been carried out in R using the
\textit{synth} and \textit{SCtools} packages, generating synthetic
Poland. Table 3.1 shows descriptive statistics of the predictors of
Poland, synthetic Poland and the donor pool mean of the predictors.
Table 3.2 presents the composition of synthetic Poland for the analysis
of the share of fully vaccinated citizens with the respective unit
weights. The data and R scripts can be found in the corresponding
\href{https://github.com/benediktstelter/bachelor_thesis.git}{GitHub repository \ExternalLink}\footnote{See appendix for further information}
\hspace{-0.1cm}.

\chapter{Results}

To begin with, a regression discontinuity design has been employed to
estimate the effect of the lottery on daily vaccinations, at the cutoff
of May 25, 2021 (announcement of the lottery).

\begin{figure}[h]
\caption{Regression discontinuity analysis of daily vaccinations in Poland}

\begin{center}\includegraphics{bachelor_thesis_files/figure-latex/unnamed-chunk-3-1} \end{center}
\end{figure}

The effect was estimated parametrically, resulting in a polynomial fit.
Figure 3.1 presents the results of the analysis. By comparing the
intercepts at the cutoff, it can clearly be seen that the lottery did
not significantly increase daily vaccinations, with the difference in
the intercepts being nearly invisible. As the figure shows, the lottery
was announced when daily vaccinations were close to their all time high,
but the polish government thought that additional motivation was
necessary. After the start of the lottery, there was a strong decrease
in daily vaccinations. Obviously, it is not possible to make any
conclusion about the causal effect based on such a correlation. Although
this regression discontinuity analysis does not suggest a significant
effect of the lottery, we need to take a closer look at vaccination
rates, in order to make any plausible conclusion on them.

\begin{figure}[h]
\caption{Synthetic control for the share of fully vaccinated in Poland}

\begin{center}\includegraphics{bachelor_thesis_files/figure-latex/unnamed-chunk-4-1} \end{center}



\begin{center}\includegraphics{bachelor_thesis_files/figure-latex/unnamed-chunk-4-2} \end{center}
\end{figure}

We therefore applied the synthetic control method, as described in
section 3.2. Figure 4.2 plots the vaccination rate (fully vaccinated)
for both Poland and synthetic Poland. We observe a solid, but not
perfect pre-treatment fit (too unspecific). After a short lag after the
intervention (as expected), we can see a slow decoupling between Poland
and its synthetic control. This difference continues increase to around
5 percentage points. As time progresses, the gap however tends to
decrease again, and by the end of November (after the end of the
lottery), the vaccination rates of Poland and synthetic Poland are back
to the same level. Simply based on the plot, one interpretation could be
that the lottery successfully induces some people who would have gotten
vaccinated anyway to do this earlier, to take advantage of the incentive
provided by the government. It however also suggests that the vaccine
lottery was not particularly effective in reaching new people. It has to
be noted that the more time has passed since the intervention, the
synthetic control becomes more unreliable, as the prediction intervals
in figure \ldots{} show. Therefore, the development in October and
November should be interpreted with caution, but at first glance, it
seems like the lottery was not successful in getting more people
vaccinated, but only in getting people vaccinated earlier.

Employing the permutation based inference techniques discussed in
section 4.1 also confirms the finding of no significant effect. Figure
\ldots{} presents the placebo study. As can be seen, the magnitude of
the effect of Poland is not comparably high and does not stand out in
any way. Actually, the effect of Poland is the least extreme of all of
the selected units. This finding also confirmed quantitatively. Using
the discussed test procedure, a \textit{p}-value of 0.7143 is obtained.
Therefore, the hypothesis that the lottery had no effect in increasing
vaccination rates cannot be rejected.

\begin{figure}[h]
\caption{Placebo plot: Poland and control countries}

\begin{center}\includegraphics{bachelor_thesis_files/figure-latex/unnamed-chunk-5-1} \end{center}
\end{figure}

\subsection*{Robustness checks}

Next, we will assess the robustness of our synthetic control. One option
is to change the time of the intervention. As discussed earlier, there
are two possible intervention points: The announcement and the start of
the lottery, with the announcement being the more sensible option. In
order to see how robust our synthetic control is, we will no use July 1,
the start of the lottery as the intervention point. As can be seen from
the upper panel of figure 4.4, changing the time of intervention from
25/05/2021 to 01/07/2021 has no visible effect on the synthetic control,
with only very minor changes in the chosen weights. Importantly, the lag
after the announcement of the lottery is clearly visible with both of
these setups, with no lag after July 1.

\begin{figure}[h]
\caption{Poland and synthetic Poland with different intervention times}

\begin{center}\includegraphics{bachelor_thesis_files/figure-latex/unnamed-chunk-6-1} \end{center}



\begin{center}\includegraphics{bachelor_thesis_files/figure-latex/unnamed-chunk-6-2} \end{center}
\end{figure}

Another way of changing the time of intervention is backdating, meaning
that the synthetic control is estimated using an arbitrary earlier
intervention time. If the synthetic control is robust, we expect no
drastic changes from the baseline result in figure 4.1. We therefore
construct a new synthetic control, with a ``fictional'' intervention
time one month earlier (April 25) than the actual investigation time. As
observable in the lower panel of figure 4.4, the changes are also not
too large, with the new synthetic control still tracking the path of
actual Poland until the actual intervention occurs relatively well. We
therefore conclude that this synthetic control is indeed robust to
changes to the time of intervention. This finding also confirms that we
do not suffer from an anticipation effect.

\begin{figure}[h]
\caption{Leave-one-out synthetic controls of Poland}

\begin{center}\includegraphics{bachelor_thesis_files/figure-latex/unnamed-chunk-7-1} \end{center}



\begin{center}\includegraphics{bachelor_thesis_files/figure-latex/unnamed-chunk-7-2} \end{center}
\end{figure}

Another possible robustness check is to leave out certain predictors or
countries. With a relatively low number of predictors and donor pool
countries, the effects of such a robustness check could potentially be
larger in this setting. Figure 4.5 presents the result of a
leave-one-out analysis of synthetic Poland, leaving out all of the
predictors once, while keeping the others in, with the same with donor
pool countries (solving the constrained minimization with only five
countries). The synthetic control is not very robust with respect to the
predictors, with three predictors being especially important to the
robustness of the synthetic control, while the three other predictors
only take on a minor weight and therefore are not as important to the
robustness of the synthetic control. The same can be said with respect
to the countries, where 3 countries received 3 larger weights in the
original synthetic control. Therefore, leaving one out of these three
countries also has a relatively large effect. Overall, the result is not
surprising, since only a total of 6 predictors and 6 donor pool
countries are used and the differences in the vaccination rates across
donor pool countries are not to be underestimated.

The results of this analysis are also in line with a synthetic control
analysis of the effect of the lottery on the vaccination rate in terms
of the first dose only, showing no signs of a significant effect of the
lottery, with a \textit{p}-value of xxxx. More details on this analysis
can be found in the appendix.

\chapter{Discussion}

What is the impact of vaccine lotteries on vaccination rates? The
results of the analysis do not suggest a significant effect of the
lottery on vaccination rates. One interpretation of the graph could be
the fact that people who would have gotten vaccinated anyways decided to
do this earlier to take advantage of the lottery.

Applying the synthetic control method in comparative case studies offers
many advantages \parencite{abadie_using_2021}. Firstly the fit of the
synthetic control is very transparent. A graph of the actual unit and
its synthetic control and a table such as table 3.1, which presents the
predictors of actual and synthetic Poland clearly show the differences
between the two units, allowing for a straightforward evaluation of
whether the use of a synthetic control is appropriate in this
application. Transparency is also an advantage with respect to the
composition of the synthetic control unit. A clear list of the different
weights provides the opportunity to assess the fulfillment of some of
the requirements of the synthetic control, e.g.~the spillover effect.
For example, the problem of a possible spillover effect could be
disregarded, when the country this might apply to has a weight of 0 or
very close to 0. Another advantage might be that only pre-intervention
outcomes are used to construct the synthetic control unit. This means
that it
might\footnote{It is of course still possible to compare the results of different specifications and select the specification which is "preferred".}
prevent researchers from changing the specifications of the synthetic
control, for example donor pool countries or covariates, to achieve a
certain result (e.g.~a significant result (\textit{p}-Hacking)), after
initially constructing it.

Besides these advantages, the synthetic control method itself also has
some disadvantages and weaknesses \parencite{bouttell_synthetic_2018}.
One advantage is that there is a lack of quantitative criteria for
crucial requirements. As we have seen in section 3.1, the similarity of
donor pool countries is a relevant criterion for a valid synthetic
control, there is however no definition of what exactly means
similarity. It is not uncommon that assumptions are hard to justify
(e.g.~instrumental variable approach (more specific)), but the argument
of similarity can be made in many directions. Another possible problem
in this respect is the judgement of the quality of the fit. There is no
objective or quantitative measure to evaluate the pre-treatment fit of a
synthetic control, meaning that the evaluation is always subject to a
possible bias of the researcher.

There are several limitations of this application of the synthetic
control method, making the estimated causal effect less valid.

Firstly, the number of donor pool countries is relatively low. Therefore
the number of combinations is obviously more limited compared to
e.g.~Ohio, where all other US States (minus states that experienced
similar policies) can be selected for the donor pool. Similarly, in the
analysis of Ohio's lottery, other (neighbouring) states are very similar
to Ohio, both in general and with respect to vaccination rates. Although
the donor pool countries in this thesis are not too different compared
to
Poland\footnote{See table 3.1 for a comparison between Poland and the average of the donor pool},
cross-country differences are still larger than cross-state differences.
Therefore the fit which is obtained from the constrained minimization is
not perfect, albeit also not particularly bad.

Secondly, a vaccine lottery might be too little of an intervention to be
relevant for a synthetic control. After all, the effect on vaccination
rates that has been estimated for similar lotteries is often relatively
low (e.g.~\textcite{barber_conditional_2022}: 1.5\%). Poland's lottery
offers more prizes, but a lower main prize than some lotteries in North
America. It might therefore be conceivable that the effect of the given
policy in Poland is too small for a causal synthetic control analysis,
since the effect of the intervention is not distinguishable from other
relatively small shocks to the vaccination rate (e.g.~some public figure
speaking out against vaccination).

Another possible limitation of this analysis are the imputed values. As
explained in section 3.2, some countries in the donor pool had a lot of
missing values, most importantly Slovakia, which only reported weekly
values and takes up the largest weight of Synthetic Poland. Although the
graphs resulting from the linear interpolation look reasonable, this
might still have a negative effect on the credibility of the presented
synthetic control, as such differences can cause changes in the chosen
weights of the synthetic control.

There are however also requirements of the synthetic control which this
application should clearly fulfill. Firstly, we do not observe a sign of
an anticipation effect, as seen in chapter 4. Secondly, the possibility
of spillover effects: We do not observe any spillover effects of this
lottery. Since we removed countries who adopted similar incentive
policies, no country in the donor pool has adopted such a policy with
Poland as a role model. Additionally, since we have shown that the
lottery did not strongly increase vaccination rates in Poland, we also
do not expect a general positive example of Poland in terms of an
increased vaccination rate on other Eastern European countries. Lastly,
the number of pre-intervention outcomes does not represent as problem.
We start the optimization on February 1, 2021 and end it one day before
the announcement, on May 24, 2021. Since daily data is used, we
therefore have a total of 113 (?) pre-intervention values of the outcome
variable, therefore meaning that a solid fit of the synthetic control
can be obtained.

To summarize, it is therefore important to note that the causal
interpretation of the given results should be done with caution, because
of these outlined limitations. This specific application is however also
not particularly bad and not one of the worst applications of the
synthetic control method (change sentence). Lastly it has to be noted
that this thesis presents only one case study. While the results do not
suggest a significant effect of the lottery on the vaccination rate, the
applicability of these results on (at first glance) comparable policies
is limited, mainly because of differences in the design of lotteries and
very large differences in initial vaccination rates and other country
specific predictors. One should therefore not make an evaluation of
vaccine lotteries based on one specific case study, but rather using the
breadth of the literature, which has been outlined in section 2.1.

\chapter{Conclusion}

The COVID-19 pandemic has brought innovative policies into action. Among
them, vaccine lotteries.

As we have seen, nudging and the use of economic incentives can be
successful in inducing changes in individual health behaviour, for
example by increasing physical activities. Lotteries and other
incentives may have also contributed to increasing vaccination rates in
the COVID-19 pandemic, for example in Ohio, whose vaccine lottery has
been evaluated several times.

In order to empirically assess the impact of vaccine lotteries on
vaccination rates in this thesis, we examine a vaccine lottery
implemented in Poland, from July 1, 2021 to September 30, 2021. The
lottery consisted of cash and non-cash prizes of around 140 million PLN.
To estimate the effect of this program on vaccination rates, the
synthetic control method was selected. A synthetic Poland was
constructed out of a donor pool of six other Eastern European countries.

The main results show no signs of a statistically significant increase
(\textit{p}-value: 0.7143) in the vaccination rate compared to the
hypothetical scenario without the lottery on the long-term. This result
is also in line with a regression discontinuity analysis of daily
vaccinations. It can however be observed that vaccination rates increase
in the short-run, thereby possibly suggesting that people who would have
gotten vaccinated anyways, may have chosen to do this earlier, as a
result of the lottery.

At the time of writing, there has been no study dealing in detail with
this specific policy in Poland. This thesis therefore adds a new case
study to the vast literature on COVID-19 vaccine lotteries, which has so
far focused on such policies in the US.

Are vaccine lotteries worth it? As discussed, the evidence on the
effectiveness is mixed. When keeping in mind the costs of such a policy,
it might even be a better idea to provide incentives at the local level,
on a smaller scale. Possibly, a beer without a flaw, is better than a
random draw, bratwurst and other snacks, outclass the ``idiot tax'', a
license for fishing, trumps a month of wishing.


 
\backmatter


 
\renewcommand\refname{References}
\printbibliography[title=References]

\chapter{Appendix}
GitHub repository: \url{https://github.com/benediktstelter/bachelor_thesis}

\noindent All of the data of the covariates and the R scripts used to analyse the data can be found in folder "scripts\_data", along with some additional information in README.



\chapter{Affidavit}
\thispagestyle{empty}

% Dieser Text entspricht den genauen Vorgaben der Richtlinien für Bachelorarbeiten sowie der Prüfungsordnung (\§ 14a) 
% Stand: November 2014
I affirm that this Bachelor thesis was written by myself without any unauthorised third-party support. All used references and resources are clearly indicated. All quotes and citations are properly referenced. This thesis was never presented in the past in the same or similar form to any examination board. 

\noindent I agree that my thesis may be subject to electronic plagiarism check. For this purpose an anonymous copy may be distributed and uploaded to
servers within and outside the University of Mannheim.

\vspace{2\baselineskip}

\noindent German translation:\\
Ich versichere, dass ich die vorliegende Arbeit ohne Hilfe Dritter und ohne Benutzung anderer
als der angegebenen Quellen und Hilfsmittel angefertigt und die den benutzten Quellen
wörtlich oder inhaltlich entnommenen Stellen als solche kenntlich gemacht habe. Diese Arbeit
hat in gleicher oder ähnlicher Form noch keiner Prüfungsbehörde vorgelegen.

\noindent Ich bin damit einverstanden, dass meine Arbeit zum Zwecke eines Plagiatsabgleichs in
elektronischer Form anonymisiert versendet und gespeichert werden kann.

\vspace{4\baselineskip}
\begin{center}
\parbox{.8\textwidth}{Mannheim, 17/03/2023 \hfill Benedikt Stelter}
\end{center}


 
\end{document}
