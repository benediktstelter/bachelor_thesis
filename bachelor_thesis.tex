% !TEX program = lualatex
% !TEX encoding = UTF-8 Unicode
% !TEX spellcheck = de_DE
% 
% Vorlage für Bachelorarbeiten
% 
% Um die Vorlage mit LaTeX zu erstellen sind folgende Programme aufzurufen:
% > lualatex hauptdatei.tex
% > biber hauptdatei
% > lualatex hauptdatei.tex

\documentclass{scrbook}

% !TEX root = hauptdatei.tex
% !TEX encoding = UTF-8 Unicode
% !TEX spellcheck = de_DE
%
% für mehr Informationen zu einzelnen Paketen siehe zum Beispiel:
% http://texdoc.org/pkg/paketname
% http://ctan.org/pkg/paketname


%% Setzen von Dokumentenoptionen:
%% (aquivalent zu \documentclass[<Optionen>]{...}

%%% Layout-Einstellungen
	\KOMAoptions{ 				% aquivalent zu \documentclass[<Optionen>]{...}
		fontsize=12pt,			% Standartschriftgröße
		bibliography=totoc,	% Bibliografie soll im Inhaltsverzeichnis auftauchen
		headings=normal,		% Größe und Abstand von Überschriften
		toc=listof,				% Verzeichnisse der Gleitumgebungen ins Inhaltsverz.
		toc=indent,				% Inhaltsverzeichnis in hierarchischer Form
		listof=indent,			% andere Verzeichnisse in hierarchischer Form
		listof=totoc,           % andere verzeichnisse im Inhaltsverzeichnis führen
		twoside=false				% enseitiges Layout
	}
	\setcounter{tocdepth}{1} % Ebenentiefe des Inhaltsverzeichnis
	\usepackage{geometry, setspace}
	\geometry{
		paper=a4paper,		% DIN A4 Papier
		hmargin=30mm,			% horizontale Seitenränder
		top=15mm,				% oberer Rand
		bottom=20mm,			% unterer Rand
		includeheadfoot,	% Kopf- und Fußzeilen gehören nicht zum Rand
	}
	\onehalfspacing 		% anderthalbfacher Zeilenabstand
	% Kapitelüberschrift etwas nach oben versetzen:
	\renewcommand*{\chapterheadstartvskip}{\vspace*{-1.48\topskip}}


%% Einstellung der Schriftart:
	\usepackage{lmodern}
	% Alternativ können mit fontspec beliebige im Betriebssystem installierte Schriften verwendet werden:
	%\usepackage{fontspec}
	%\setmainfont{Constantia}
	%\setsansfont{Corbel}
	%\setmonofont{Consolas}
	% Für Serifen in den Überschriften:
	%\addtokomafont{sectioning}{\rmfamily}


%% Einstelen von Kopf- und Fußzeilen:
	\usepackage[headsepline=false]{scrlayer-scrpage}
	% automatisches Füllen der Kopfzeile mit aktuellem Kapitel/Abschnitt:
	\automark[chapter]{chapter}
	% links, mitte, rechts
	\lohead{}
	\cohead{}
	\rohead{}
	% Seitenzahl nur auf plain-Seiten im Fuß
	\lofoot{}
	\cofoot*{\pagemark}
	\rofoot{}
	% Aktivieren des festgelegten Kopfzeile:
	\pagestyle{scrheadings}


%% Sprachauswahl:
	\usepackage{polyglossia}
	\setmainlanguage{english}
	% die Sprache kann im Dokument mit
	% \begin{english} ... \end{english}
	% umgestellt werden


%% Einstellungen zu Zitaten und Bibliografie:
\usepackage{csquotes}

\usepackage[
backend=biber,
bibstyle=apa,
citestyle=apa,
]{biblatex}


\setlength{\bibhang}{15pt}
\defbibenvironment{bibliography}
  {\list
     {}
     {\setlength{\leftmargin}{\bibhang}%
      \setlength{\itemindent}{-\leftmargin}%
      \setlength{\itemsep}{\bibitemsep}%
      \setlength{\parsep}{\bibparsep}}}
  {\endlist}
  {\item}
  \addspace

\setlength\bibitemsep{1.3\itemsep}


	%% sonstige Pakete:
	\usepackage{
		array,		% erweiterte Option für Tabellen
		booktabs,	% schöne Tabellen
		float,		% Platzierung von Gleitobjekten (Abb., Tab.), Eigene Gleitobjekte
		graphicx,	% ermöglicht einbinden von Grafiken mit \includegraphics
		hologo,		% für TeX-Logos
		mathtools,	% Verbesserungen für den Mathesatz (läd u.a. amsmath)
		microtype,	% mikrotypografische Verbesserungen (z.B. optischer Randausgleich)
		paralist,	% platzsparende Listen mit compactitem
		xcolor,		% Verwendung von Farbe
	}


%% LaTeX sucht nach Bildern an den hier angegebenen Stellen:
	\graphicspath{{./images/},{./}}


%% automatische PDF-Verlinkungen im Dokument:
	\usepackage[
		colorlinks=false,		% Links nicht farbig hervorheben
		pdfborder={0 0 0},		% links nicht durch PDF-Kasten hervorheben
	]{hyperref}






\addbibresource{thesis.bib}


\usepackage{tikz}

\usepackage{array,makecell}
\usepackage[format=plain, labelfont=bf]{caption}


\usepackage{tabularx}
    \renewcommand\tabularxcolumn[1]{m{#1}}
    \newcolumntype{C}{>{\centering\arraybackslash}X}

\newcommand{\ExternalLink}{%
    \tikz[x=1.2ex, y=1.2ex, baseline=-0.05ex]{% 
        \begin{scope}[x=1ex, y=1ex]
            \clip (-0.1,-0.1) 
                --++ (-0, 1.2) 
                --++ (0.6, 0) 
                --++ (0, -0.6) 
                --++ (0.6, 0) 
                --++ (0, -1);
            \path[draw, 
                line width = 0.5, 
                rounded corners=0.5] 
                (0,0) rectangle (1,1);
        \end{scope}
        \path[draw, line width = 0.5] (0.5, 0.5) 
            -- (1, 1);
        \path[draw, line width = 0.5] (0.6, 1) 
            -- (1, 1) -- (1, 0.6);
        }
    }


\usepackage{booktabs, caption, makecell}
\renewcommand\theadfont{\bfseries}
\usepackage{threeparttable}

\counterwithout{footnote}{chapter}


%% Alle wichtigen Einstellungen sind in der Datei einstellungen.tex getätigt
%% und können dort verändert werden.

\begin{document}
	
\frontmatter
\begin{titlepage}

\begin{center}

\vspace*{1,2cm}

\huge {\bfseries The impact of vaccine lotteries on COVID-19 vaccination
rates: Evidence from Poland}\\[1.8cm]

\Large {Bachelor Thesis}\\[1cm]

\large {Department of Economics}\\[0.2cm]

\large {University of Mannheim}\\[0.5cm]

\end{center}

\vfill

\noindent submitted to:\\
Prof.~Achim Wambach, PhD / Sabrina Schubert\\[1cm]
submitted by:\\
Benedikt Stelter\\[1cm]
Student ID: 1731015\\
Degree Programme: Bachelor of Science in Economics (B.Sc.)\\[1cm]
Address: Meerfeldstr. 11, 68163 Mannheim\\
Phone: +49 176 95741248\\
E-Mail: benedikt.stelter@students.uni-mannheim.de\\[1cm]
Mannheim, 17/03/2023

\setcounter{page}{0}

\end{titlepage}

  \tableofcontents


%% Abbildungsverzeichnis
\listoffigures

%% Tabellenverzeichnis
\listoftables


\mainmatter

\chapter{Introduction}

In March 2020, the world came to a sudden stop due to the spread of
COVID-19. Governments took drastic and unprecedented steps to slow the
spread of the virus: Shops were shut down, schools and universities were
closed and millions of workers had to work from their home office or
kitchen desk.

Just a year after the first COVID-19 cases appeared in China, the first
vaccines against COVID-19 were approved. With them came the hope of a
return to normality and a severe reduction of the death toll of the
pandemic. At the beginning of the vaccination effort in Europe, the
supply of vaccines was not able to meet the demand at the necessary
speed and scale \parencite{bongardt_europes_2021}. Therefore, a shortage
of vaccines lead to rationing: Vaccines were only provided to the most
vulnerable groups of the population, such as healthcare workers or the
elderly. When vaccines became widely available, policymakers had to
learn that a considerable share of people were hesitant or unwilling to
receive the vaccine \parencite{steinert_covid-19_2022}. Globally, the
COVID-19 vaccine was one of the most relevant and hotly contested topics
of 2021, demonstrated by Google
Trends\footnote{see \url{https://trends.google.de/trends/yis/2021/GLOBAL/}},
which placed the COVID vaccine as the 3rd most searched news story of
the entire year of 2021.

The problem of vaccine hesitancy led governments to think in new ways
again, with many deciding to offer various sorts of incentives to get
vaccinated \parencite{wyllie_jewellery_2023}. One of these incentives
were vaccine lotteries. A vaccine lottery, in the context of this
thesis, refers to a lottery with cash or non-cash prizes, in which
vaccinated people could participate at no additional cost, thereby
acting as a possible reward for vaccination. Policymakers hoped that
this would persuade additional people to get vaccinated, increasing the
total vaccination rate of the population.

Several lotteries of this sort were implemented around the world, whose
effects have been widely investigated. \textcite{dube_exploring_2022}
analysed the effectiveness of a vaccine lottery in Québec (Canada) and
found a small positive impact on vaccination rates. A survey, conducted
by \textcite{jun_association_2022}, concluded that a vaccine lottery in
Australia successfully increased willingness to get vaccinated. Several
lotteries have also been implemented in the US. Studies suggest that
most but not all of the lottery programs in US states were successful in
increasing vaccine uptake (\cite{robertson_are_2021};
\cite{acharya_implementation_2021}; \cite{fuller_assessing_2022}). There
have been three vaccine lotteries implemented in Europe, in Romania,
Slovakia and Poland. Poland was the first European country to implement
such a policy and offers good preconditions (data availability, detailed
information about the lottery) to investigate the effect of such a
policy.

In order to spur the national vaccination campaign, the Polish
government announced the lottery
\parencite{service_of_the_republic_of_poland_national_2021} on May 25,
2021, around five months after the authorisation of the first COVID-19
vaccines in the EU and the subsequent start of the Polish vaccination
campaign. It was open from July 1, 2021 until September 30, 2021. The
main prize of the lottery was a cash prize of one million zł
(zloty)\footnote{At the time of announcement on 25/05/2021, one million zł were equal to around 220,000 €, at an exchange rate of 0.22 zł to €},
but it also included smaller cash and non-cash prizes and a lottery-like
incentive scheme for municipalities.

Among with other vaccination incentives in Europe,
\textcite{kuznetsova_effectiveness_2022} shortly evaluated the effects
of this lottery on daily vaccinations using a time-series model,
suggesting a positive effect of the policy. To this day, there has
however been no study evaluating the Polish vaccine lottery in detail.
This thesis therefore adds a new case study to the existing literature
on the effects of COVID-19 vaccine lotteries.

The quasi-experimental empirical analysis consists of two parts. To
begin with, a regression discontinuity design was employed to evaluate
the effect of the lottery on daily vaccinations, similar to the analysis
of \textcite{kuznetsova_effectiveness_2022}. The objective behind a
regression discontinuity design is to exploit the existence of a
cutoff/threshold value. Using the day of announcement as the cutoff
value, the effect of the lottery is estimated. The main analysis is
conducted using the synthetic control method based on
\textcite{abadie_economic_2003}, a popular method for comparative case
studies. The idea behind the synthetic control method is to create a
synthetic control unit as a combination of multiple control units (donor
pool), to then estimate the average treatment effect of the
policy/intervention. Out of a donor pool of six other Eastern European
countries and using additional predictor variables, a synthetic control
unit (``Synthetic Poland'') is constructed to evaluate the effect of the
policy on vaccination rates of both fully vaccinated and people with at
least one dose. Data on daily vaccinations and vaccination rates (both
fully vaccinated and at least one dose), published by the respective
governments/government agencies and collected by Our World in Data
\parencite{mathieu_global_2021} as well as additional data from
\textcite{eurostat_eurostat_2023} and the Wellcome Global Monitor,
collected by \textcite{our_world_in_data_share_2020}, are exploited for
the analysis.

The empirical analysis finds no evidence for a significant increase in
the share of the population vaccinated against COVID-19. It suggests
that the lottery might have incentivized some people who would have
gotten vaccinated anyway to do this earlier, to take advantage of the
lottery, thereby potentially increasing the vaccination rate temporarily
by around five percentage points.

Chapter 2 provides background information about the existing literature
and the policy. Chapter 3 discusses the methodology and the data used
for the empirical analysis. The results are presented in Chapter 4 and
discussed in Chapter 5.

\chapter{Background}

\section{Literature review}

\subsection*{Incentives in public health}

Tobacco and alcohol consumption, obesity, viral diseases: Governments
are facing many challenges from a public health perspective. Behavioural
economics can offer several ways to influence the decisions of
individuals, in order to encourage socially desirable behaviour. The
least severe form are nudges.

Nudging is a concept mainly brought to the public through the work of
\textcite{thaler_nudge_2008}. The authors define a nudge as an
intervention that ``alters people's behavior in a predictable way
without forbidding any options or significantly changing their economic
incentives'' (p.~6). Nudging can be applied in various ways, but it is
often connected to health lifestyle topics, such as nutrition and diet
\parencite{ledderer_nudging_2020}. For example, in an experiment in
Denmark \parencite{friis_comparison_2017}, the objective was to promote
the consumption of vegetables in the setting of a self-serving buffet,
which included salads and other dishes in large bowls. As a nudge, the
food environment was changed by arranging green plants and herbs around
the food bowls. In a second experiment, salad was pre-portioned into
smaller take-away bowls. The results showed that the intake of energy
from vegetables of the participants can be increased by pre-portioning
the salad. Although many studies evaluating the impact of nudges find
positive effects of the respective interventions, these results should
be interpreted with caution, since a lot of studies, including
\textcite{friis_comparison_2017}, were conducted in the lab and may
therefore not be reproducible in a real-world setting
\parencite{ledderer_nudging_2020}.

Besides nudging, policymakers can significantly change economic
incentives. This has especially been applied in drug policy, for example
regarding alcohol and tobacco consumption. Both in developed and
developing countries, it has been shown that raising prices (through
higher taxation) can lead to reduced consumption of tobacco
(\cite{yeh_effects_2017}; \cite{immurana_effects_2021}) as well as
alcohol \parencite{daley_impact_2012}.

Beyond taxation, governments could also use other financial incentives
to motivate changes in individual behaviour, such as (small) cash
payments or lotteries. Several meta-analyses found that such incentives
can be successful in inducing behaviour changes at the individual level.
For example, \textcite{giles_effectiveness_2014} evaluated 16 studies on
issues such as smoking cessation, health screenings, physical activity
and vaccinations. The authors found that financial incentives are
effective at encouraging healthy behaviour. This finding was also
confirmed by \textcite{mantzari_personal_2015}, who evaluated 34 studies
and additionally concluded that this effect is stronger for the most
deprived individuals, thereby possibly reducing health inequalities.
Lotteries have also been shown to be effective in certain public health
settings. \textcite{bjorkman_nyqvist_incentivizing_2018} found that the
introduction of a lottery program reduced HIV incidence in Lesotho.
Lotteries also increased cycling \parencite{ciccone_using_2021} and
walking activity \parencite{patel_randomized_2018} as well as
participation in chlamydia screenings \parencite{niza_vouchers_2014}
under specific settings.

\subsection*{Incentives to increase COVID-19 vaccination rates}

When a considerable degree of citizens were unwilling to get vaccinated
in Europe \parencite{steinert_covid-19_2022}, governments thought about
ways to encourage their citizens to get vaccinated, using incentives as
well as other measures.

A very widely used tool were vaccine passports (e.g.~in Europe
\parencite{niestadt_domestic_2021}. Access to social gatherings
(e.g.~restaurants, bars, clubs, stadiums), international travel and
quarantine regulations was made subject to certified vaccination against
COVID-19. Besides ethical concerns, for example potential
disqualification of minorities from social life
\parencite{gostin_digital_2021}, the evidence on the effectiveness of
such regulations is mixed. A synthetic control analysis of six countries
found that COVID passports were successful in increasing daily
vaccinations in countries with lower than average vaccination rates. In
other countries these regulations were less effective
\parencite{mills_effect_2022}. In Eastern Europe, a comparison between
Poland (very restricted use of COVID passports) and Lithuania (wide use
of COVID passports) suggested a positive effect on vaccination rates
\parencite{walkowiak_covid-19_2021}. It should be noted that COVID
passports may also have negative effects on vaccine uptake, since
frustration about reduced autonomy might lower willingness to get
vaccinated \parencite{porat_vaccine_2021}.

Besides the carrot-and-stick approach, governments have also used nudges
and economic incentives, such as cash payments and non-cash rewards to
increase COVID-19 vaccination rates. Using a randomised control trial,
\textcite{campos-mercade_monetary_2021} found that even ``small'' cash
payments of around 24 US dollars can significantly increase vaccination
rates, while small nudges, such as information about the safety and
effectiveness of the vaccine, were not successful. While some studies
also suggest a measurable positive effect of cash payments on
vaccination rates (\cite{wong_guaranteed_2022};
\cite{kluver_incentives_2021}; \cite{kim_vaccination_2021-1}), there is
also evidence against its effectiveness. \textcite{jacobson_can_2022}
suggested that neither behavioural nudges (reminder to get vaccinated
via text message) nor cash payments could increase vaccination rates
among the hesitant citizens. \textcite{sprengholz_money_2021} also found
that cash incentives did not increase willingness to get vaccinated.

A special version of economic incentives are lotteries. The majority of
studies on COVID-19 vaccine lotteries dealt with lotteries in the US.
Studies on vaccine lotteries in Louisiana and Massachusetts found
different effects. Whereas the lottery in Louisiana increased vaccine
uptake \parencite{wang_moving_2023}, a vaccine lottery in Massachusetts
did not significantly increase vaccination rates, although prizes were
higher \parencite{kim_did_2023}. When evaluating several state vaccine
lotteries, the available evidence suggests that most, but not all of
these policies were successful in increasing vaccine uptake
(\cite{robertson_are_2021}; \cite{acharya_implementation_2021};
\cite{fuller_assessing_2022}).

A specific focus can be observed with respect to Ohio. There are several
studies evaluating the Ohio Vax-A-Million lottery
\parencite{ohio_department_of_health_ohio_2021}, which was one of the
first COVID-19 vaccine lotteries in the US. In total, a majority of the
reviewed literature casts a positive light on the efficacy of the
lottery. There have been four studies evaluating the lottery using the
synthetic control method. These studies constructed a synthetic Ohio out
of a donor pool of other US states. Three of these four studies found
small positive effects of the lottery on vaccine uptake
(\cite{brehm_ohio_2022}; \cite{barber_conditional_2022};
\cite{sehgal_impact_2021}) and one did not find a significant effect
\parencite{lang_did_2022}. Regarding studies using other methods,
\textcite{mallow_covid-19_2022} also found a positive effect of the
lottery on vaccination rates, while \textcite{walkey_lottery-based_2021}
did not find such an effect.

\section{Institutional background}

The Polish vaccination campaign started on December 27, 2020, when a
nurse from a Warsaw Hospital received the first vaccine dose
\parencite{waligora_how_2021}. Until September
2021\footnote{Vaxzevria and Spikevax were no longer used}, all of the
vaccines approved by the EU (Vaxzevria (AstraZeneca), Comirnaty
(Pfizer/BioNTech), Spikevax (Moderna) and Janssen (Johnson \& Johnson))
were used for vaccinations. Citizens who wanted to get vaccinated had to
register and select their preferred vaccine prior to the vaccination. In
May 2021, the waiting time between the first and the second dose
(Janssen required only one dose) was set at five weeks. In the first
weeks and months of the vaccination campaign, vaccines were only
available to health care workers and senior citizens. By the start of
May 2021, vaccines became widely available
\parencite{koschalka_poland_2021}.

In general, vaccination rates in Poland have been low compared to the
rest of the EU, with the Polish vaccination rate at around 50\% by the
end of October 2021, compared to an EU average of around 65\%. In the
context of Eastern Europe, vaccination rates have been relatively close
to the average. Many Eastern European countries experienced relatively
low vaccination rates \parencite{mathieu_global_2021}. In the media, a
general distrust into the government and a lack of educational campaigns
\parencite{vaccines_today_polands_2021} as well as chaotic and
conflicting communication by government officials
\parencite{wanat_polands_2021} have been cited as potential reasons for
the low vaccine uptake in Poland.

\renewcommand*{\arraystretch}{1.5}
\begin{table}[! htbp]\centering \caption[Prizes of Polish lottery]{The Polish vaccine lottery consisted of four types of draws and included cash and non-cash prizes}
\bigskip
\label{table:summarystat}
\begin{threeparttable}
\begin{tabularx}{10.5cm}{c|c|c}
\toprule\midrule
 & \thead{Cash prizes} & \thead{Non-cash prizes}\\ \midrule
Instant & \(13,000*500\) zł & - \\
 & \(39,000*200\) zł & \\ \hline
Weekly & \(60*50,000\) zł & 720 electric scooters \\  \hline
Monthly & \(6*100,000\) zł & 6 small vehicles \\ \hline
Main & \(2*1,000,000\) zł & 6 middle class vehicles \\
\bottomrule
\end{tabularx}
\begin{tablenotes}
      \item \footnotesize Source: \textcite{service_of_the_republic_of_poland_national_2021}
    \end{tablenotes}\end{threeparttable}
\label{table2}
\end{table}

\renewcommand*{\arraystretch}{1}

To increase vaccination rates, Poland decided to implement a vaccine
lottery. The lottery, which is investigated empirically in this thesis,
was open from July 1, 2021 to September 30, 2021. It was announced on
May 25, 2021 \parencite{charlish_poland_2021}. The policy had two main
elements: A lottery for all adult fully vaccinated people (two doses)
\parencite{service_of_the_republic_of_poland_national_2021} and a
monetary incentive scheme for municipalities
\parencite{service_of_the_republic_of_poland_competitions_2021}. The
main prize of the lottery was a cash prize of one million zł (awarded
twice), but it also included smaller monthly, weekly and daily cash
prizes along with non-cash prizes (cars and electric scooters),
amounting to a total cost of roughly 140 million zł
\parencite{wilczek_poland_2021}. A detailed overview of the lottery is
depicted in Table 2.1. Citizens were able to enter the lottery both
online and by phone. It was organized by the state-owned polish lottery
company Totalizator Sportowy, which also operates other popular
lotteries in Poland. Poland's lottery can be seen as a mixture of
different types of lotteries. Especially in the US, state governments
have focused on lotteries with high rewards and relatively low winning
probabilities (e.g.~Ohio, Massachusetts with prizes of one million
dollars). Another possibility could be the use of relatively low prizes
(e.g.~below 1000 dollars) with higher winning probabilities. The Polish
policy included both relatively small prizes (instant prizes) but also
larger prizes (main draw), thereby combining both elements. The vaccine
lottery to be considered as the most similar to Poland's is the one
implemented in Romania, in October 2021, which also included a similar
mixture of larger and smaller prizes
\parencite{health_ministry_of_romania_press_2021}.

As part of the monetary incentive scheme for municipalities, the
municipality with the highest percentage of the vaccinated in the
country received two million zł. Three other municipalities who had the
highest percentage of the vaccinated in their respective comparison
group\footnote{There were three groups: Municipalities with a population of up to 30,000 people, cities with a population of 30,000 - 100,000 and large cities with a population above 100,000.}
received one million zł each. 500 other municipalities, who were among
the fastest in the country to reach a vaccination rate of 67\%, won
100,000 zł each.

\chapter{Methods and data}

\section{Regression discontinuity design}

Regression discontinuity designs are a method to estimate treatment
effects in settings in which the treatment assignment is determined by
whether the running variable \(X_{i}\) exceeds a certain
cutoff/threshold value \(c\). Formally, the treatment status is defined
as: \begin{equation}
D_{i} = \begin{cases}
1 & \text{if}\; X_{i} \geq c \\
0 & \text{if}\; X_{i} < c
\end{cases}
\end{equation} In the application of the Polish lottery, the running
variable \(X_{i}\) refers to the date and the cutoff value \(c\) to the
announcement date of the lottery (May 25, 2021). Treatment therefore
occurs once the lottery is announced. This specific application is an
example of a sharp regression discontinuity design, in which the
probability of treatment changes from 0 to 1 at the cutoff. The
discontinuity of the treatment status around the cutoff allows for an
examination of the two sides, by comparing the limits at the cutoff
value. The treatment effect for the outcome variable \(Y_{i}\) at the
cutoff is defined as the difference between these limits:
\begin{align}    
\tau & =E[Y_{1}-Y_{0}\vert X_{i}=c] \\
     & =\lim_{x\downarrow c} E[Y_{1}\vert X_{i}=c]-\lim_{x\uparrow c} E[Y_{0}\vert X_{i}=c] \nonumber
\end{align} \noindent In the application of the Polish lottery, the
effect is estimated using a polynomial form. The identifying equation
for the estimation, in which the specific functional form is left open,
is given by: \begin{equation}
Y_{i}=f(X_{i})+\beta D_{i}+\epsilon_{i}
\end{equation} This can also be adjusted two allow for different
functions to the left and to the right of the cutoff, which will be
employed in this thesis: \begin{equation}
Y_{i}=f_{l}(X_{i})I\{X_{i}<c\}+f_{r}(X_{i})I\{X_{i}\geq c\}+\beta D_{i}+\epsilon_{i}
\end{equation}

\section{Synthetic control method}

The synthetic control method was first established by
\textcite{abadie_economic_2003} in order to investigate the economic
effects of terrorism in the basque country and further developed by
\textcite{abadie_synthetic_2010}. \textcite{abadie_using_2021} is a
summary of these prior developments as well as the requirements and
inference methods and can be regarded as the basis of this section.

Synthetic control methodology has been applied widely in economics, but
also in other social sciences such as political science. For instance,
it has been used to evaluate the economic effects of European
integration \parencite{campos_institutional_2019} or the effect of
natural disasters on economic growth
\parencite{cavallo_catastrophic_2013}, by constructing synthetic control
countries without European integration or natural disasters. It has also
been used to investigate democratic backsliding
\parencite{meyerrose_unintended_2020}. Its area of application are
comparative case studies, in which the comparison of two different cases
can allow for conclusions about a certain policy or intervention, which
might be present only in one of the two examined cases.

\subsection*{Formal definition}

Researchers want to investigate the effect of a policy or intervention
on a selected variable. The variable of interest is the outcome variable
\(Y_{jt}\) for \(J + 1\) units from \(t=1\) to \(T\). The first unit
(\(j = 1\)) is the treated unit (treatment occurs for \(t>T_{0}\)),
while all other units are untreated. The intervention (in this analysis
the announcement of the lottery) occurs at \(T_{0}+1\), meaning that
there are \(T_{0}\) pre-intervention time periods. There are a total of
\(k\) predictors per unit, which include the pre-intervention
observations of the outcome variable (\(Y_{j1},Y_{j2},...,Y_{jT_{0}}\))
as well as additional time invariant unit level characteristics
\(\mathbf{Z}_{j}\) (as outlined in Section 3.2). These \(k\) total
predictors can be summarized by the \((k\times 1)\) vectors
\(\mathbf{X}_{j}=(Y_{j1},Y_{j2},...,Y_{jT_{0}},\mathbf{Z}_{j}')'\) for
units \(j=1,...,J + 1\). It is then possible to combine all of the
predictors of the untreated units, in order to obtain the
\((k\times J)\) matrix
\(\mathbf{X}_{0}=(\mathbf{X}_{2},\mathbf{X}_{3},...,\mathbf{X}_{J + 1})\)\footnote{\(\mathbf{X}_0=
\begin{bmatrix}
Y_{21} & Y_{31} & \dots & Y_{J+11}\\
Y_{22} & Y_{32} & \dots & Y_{J+12}\\
\vdots & \vdots & \ddots & \vdots\\
Y_{2T_{0}} & Y_{3T_{0}} & \dots & Y_{J+1T_{0}}\\
\mathbf{Z}_{2}' & \mathbf{Z}_{3}' & \dots & \mathbf{Z}_{J + 1}'
\end{bmatrix}\)}.

The average treatment effect is defined as the difference between the
potential outcome of the treated unit with intervention (\(Y_{1t}^{I}\))
and its potential outcome without the intervention (\(Y_{1t}^{N}\)):
\begin{equation}
\tau_{1t}=Y_{1t}^{I}-Y_{1t}^{N}\; \; \text{for}\; \; t>T_{0}
\end{equation} By definition, for \(t>T_{0}\), the outcome with
intervention is known and the outcome without intervention is
hypothetical for the treated unit. To estimate the treatment effect, it
is therefore sufficient to estimate \(Y_{1t}^{N}\).

The simplest idea to estimate \(Y_{1t}^{N}\) might be the use of a
difference-in-differences with matching approach. One could choose the
closest unit \(j^{*}\), the best single control
\parencite{doudchenko_balancing_2016}, as the control unit. This best
single control solves: \begin{equation}
j^{*}=\text{arg}\; \min_{j>1}\vert\vert\mathbf{X}_{j}-\mathbf{X}_{1}\vert\vert
\end{equation} Taking a difference then results in an estimator for the
average treatment effect: \begin{equation}
\hat{\tau}_{1t}=Y_{1t}^{I}-Y_{j^{*}t}
\end{equation} Using a single control approach does not seem like a
desirable estimation method. Firstly, in many cases (including the
Polish lottery) a single control does not achieve a good pre-treatment
fit, especially when cross-unit differences are relatively large.
Secondly, a difference-in-differences approach requires the parallel
trend assumption, which might not hold in this setting. Parallel trend
requires that without treatment, the difference between treatment and
control group does not change over time. In the application of the
Polish vaccine lottery, this means that the difference in the
vaccination rates of Poland and country \(j^{*}\), the closest unit
pre-intervention, would have to be be constant if there were no lottery.
Because of the large number of country specific factors influencing
vaccination rates, changes in the gap between treatment and control unit
seem relatively likely, thereby potentially violating the assumption.

The synthetic control method developed by
\textcite{abadie_economic_2003} proposes to use a weighted average of
donor pool units as a synthetic control unit, thereby not experiencing
the problems of a single control estimation (synthetic control does not
require parallel trend assumption). The synthetic control and the
following estimator for the average treatment effect are defined as:
\begin{equation}
\hat{Y}_{1t}^{N}=\sum_{j=2}^{J+1} w_{j}Y_{jt}
\end{equation} \begin{equation}
\hat{\tau}_{1t}=Y_{1t}^{I}-\hat{Y}_{1t}^{N}\; \; \text{for}\; \; t>T_{0}
\end{equation} The weights \(\mathbf{W}=(w_{2},w_{3},...,w_{J+1})'\) are
chosen, such that the synthetic control matches as closely as possible
the pre-intervention path of the predictors of the outcome variable for
the treated unit. Consequently, the weights have to be chosen such that
this difference is minimized. The optimal weights
\(\mathbf{W}^{\star}=(w_{2}^{\star},w_{3}^{\star},...,w_{J+1}^{\star})'\)
solve:

\begin{equation}   
\begin{aligned}
\mathbf{W}^{*}=\text{arg}\; \min \quad & \vert\vert\mathbf{X}_{0}\mathbf{W}-\mathbf{X}_{1}\vert\vert\\
\textrm{s.t.} \quad & w_{j}\in[0,1]\\
  &\sum_{j=2}^{J+1} w_{j}=1   \\
\end{aligned}
\end{equation} This is subject to the weights being non-negative and
summing up to one, a crucial assumption in the original framework
proposed by \textcite{abadie_economic_2003}. In several extensions to
this original framework (e.g.~\textcite{doudchenko_balancing_2016}),
this assumption has been relaxed to allow for negative weights, it will
however be kept in this thesis. It is then possible to plug in the
optimal weights \(\mathbf{W}^{\star}\) from the constrained minimization
to obtain the estimated average treatment effect from (3.x):
\begin{equation}
\hat{\tau}_{1t}=Y_{1t}^{I}-\sum_{j=2}^{J+1} w_{j}^{\star}Y_{jt}\; \; \text{for}\; \; t>T_{0}
\end{equation}

\subsection*{Inference}

Based on \textcite{abadie_synthetic_2010}, the most common way of
inference in synthetic control is permutation, through the use of
placebo effects. A synthetic control unit is constructed for all the
untreated countries in the control group, as if there was a treatment
for these countries. If the magnitude of the effect for the treated unit
is extreme compared to the placebo synthetic controls of the untreated
units, the effect can be regarded as significant. In order to perform
this analysis, the gaps between the synthetic control and the actual
outcome can be plotted for all selected countries (``placebo plot''), in
order to visually compare the size of the effects. One potential problem
of this concept is the difficulty of obtaining a satisfactory
pre-treatment fit for all units in the donor pool, especially when
making use of a relatively small donor pool. Additionally, inference
based on visual interpretations can be considered as vague, since it
does not rely on a quantitative measure, such as a \textit{p}-value.

A possibility of quantification is to measure the ratio of the
post-intervention fit relative to the pre-intervention fit. First of
all, the root mean squared prediction error (RMSPE) of the synthetic
control is defined: \begin{equation}
R_{j}(t_{1},t_{2}) =(\frac{1}{t_{2}-t_{1}+1}\sum_{t=t_{1}}^{t_{2}}(Y_{jt}-\hat{Y}_{jt}^{N})^2)^\frac{1}{2}
\end{equation} It is then possible to compute \(r_{j}\), which measures
the quality of the fit in the post-intervention period compared to
pre-intervention and is given by the ratio of the post-intervention
RMSPE and pre-intervention RMSPE: \begin{equation}
r_{j}=\frac{R_{j}(T_{0}+1,T)}{R_{j}(1,T_{0})}
\end{equation} A p-value can then be computed for the permuted test:
\begin{equation}
p=\frac{1}{J+1}\sum_{j=1}^{J+1}I_{+}(r_{j}-r_{1})
\end{equation} where \(I_{+}(\cdot)\) is an indicator function that
returns one for non-negative arguments and zero otherwise.

\subsection*{Requirements}

Several requirements should be fulfilled in order to obtain a valid
synthetic control estimation, outlined by \textcite{abadie_using_2021}.
Firstly, the evaluated policy/intervention should - in principle - be
able to produce a sufficiently large effect. When the effect of an
intervention is too small, it may not be possible to distinguish this
effect from other shocks to the outcome variable. Additionally, the
volatility of the outcome variable should not be too high, to prevent
``over-fitting'', meaning that the synthetic control might react to
certain pre-treatment patterns of the outcome variable which could not
be present post treatment \parencite{hollingsworth_tactics_2022},
generating a biased estimation.

Secondly, there needs to be a suitable comparison/control group.
Countries that are also subject to similar interventions or other shocks
to the outcome variable in the given time frame, should be excluded from
the donor pool. If this were not done, a negative shock to the outcome
variable in one of the donor pool countries could lead to false
conclusions. It could be mistakenly determined that the country of
interest experienced a negative effect because of the examined
intervention, which might only be the case because of the shock in one
of the donor pool countries. Furthermore, researchers should try to
select countries for the donor pool which are not too different from the
treated country. Otherwise, it might be difficult to obtain a good fit
of the synthetic control unit before the intervention.

Thirdly, it is crucial not to be subject to an anticipation effect,
meaning that economic agents anticipate the enactment of a policy. In
contrast, if agents were to anticipate the intervention before its
actual implementation, the synthetic control analysis could not
successfully estimate the investigated effect. Too many pre-treatment
observations would then be used for the constrained minimization, which
generates the weights of the synthetic control unit.

Next, there should be no spillover effects of the treated unit on
untreated units. If this were the case, the post-treatment values of the
synthetic control unit (which, by definition, consists of untreated
units) would be affected by the treatment, thereby violating the basic
principle of the synthetic control method.

Lastly, regarding the data, it is crucial to have a sufficient number of
pre-treatment observations of the outcome variable, so that a
satisfactory fit of the synthetic control unit can be obtained. A larger
number of pre-intervention outcomes improves the fit of the synthetic
control unit, increasing its robustness and interpretability.

\subsection*{Strengths and weaknesses}

Applying the synthetic control method in comparative case studies offers
several advantages \parencite{abadie_using_2021}. Firstly the fit of the
synthetic control is transparent. A graph of the actual unit and its
synthetic control and a table such as Table 3.1, which presents the
predictors of actual and Synthetic Poland, outline the differences
between the two units, allowing for a quick evaluation of whether the
use of a synthetic control is appropriate in this application.
Transparency is also an advantage with respect to the composition of the
synthetic control unit. A clear list of the different weights provides
the opportunity to assess the fulfillment of some of the requirements of
the synthetic control, e.g.~the spillover effect. The problem of a
possible spillover effect could be disregarded, if the country this
might apply to has a weight of 0 or very close to 0. Another advantage
might be that only pre-intervention outcomes are used to construct the
synthetic control unit. This could prevent researchers from changing the
specifications of the synthetic control to achieve a certain result
(e.g.~a significant result (\textit{p}-Hacking)), after initially
constructing it. Still, it has to be noted that it is possible to
compare the results of different specifications and select the
specification which is ``preferred'', but the described advantage might
make this more unlikely.

Besides these advantages, the synthetic control method itself also has
some disadvantages and weaknesses \parencite{bouttell_synthetic_2018}.
One disadvantage is that there is a lack of quantitative criteria for
crucial requirements. As seen in Section 3.1, the similarity of donor
pool countries is a relevant criterion for a valid synthetic control,
there is however no clear definition of what exactly constitutes
similarity. It is not uncommon that assumptions leave room for differing
interpretations in econometric applications (e.g.~exclusion restriction
of an instrument in IV approach), but the argument of similarity can be
made in many directions. Another possible problem is the judgement of
the quality of the fit. There is no objective measure to evaluate the
pre-treatment fit of a synthetic control unit, meaning that the
evaluation is always subject to a possible bias of the researcher.

\section{Data and approach}

First, a regression discontinuity design is employed to evaluate the
effect of the lottery on daily COVID-19 vaccinations in Poland, similar
to the analysis by \textcite{kuznetsova_effectiveness_2022} who also
investigated this effect.

The data for this analysis as well as the main synthetic control
analysis, which investigates the impact of the policy on the vaccination
rates, is taken from a data set created by Our World in Data
\parencite{mathieu_global_2021}. This data set is a collection of the
number of daily vaccinations, vaccination rates along with other
vaccination-related indicators from all countries of the world, coming
directly from the respective government/government agency and collected
by Our World in Data. If provided by the governments, the data set
offers a daily time series of the described indicators.

One of the most important aspects in the practical application of the
synthetic control method is the choice of the donor pool. As outlined in
Section 3.1, the treated country should not be an outlier compared to
its control units. It is therefore sensible to select countries that are
similar to Poland, both in general and, most importantly, with respect
to their vaccination rates. Eleven Eastern European countries (BG, CZ,
EE, GR, LV, LT, HR, HU, RO, SI, SK) were initial candidates for the
donor
pool\footnote{Austria was also considered, but its vaccination rate is high compared to the other potential donor pool countries and the general differences to Poland are relatively large (e.g. in terms of GDP per capita).},
because of the resemblance in vaccination rates as well as a
similarities in other variables (see Table 3.1). From this list, several
countries are dropped, who implemented similar interventions or
experienced other shocks to the vaccination rate in the given time
frame: Greece (cash
incentive\footnote{see \textcite{koutantou_greece_2021}}), Slovakia
(lottery incentive\footnote{see \textcite{lopatka_slovaks_2021}}) and
Romania (lottery
incentive\footnote{see \textcite{health_ministry_of_romania_press_2021}}).
Estonia, Czechia and Slovenia implemented some small incentive schemes
for general practicioners (Estonia and
Slovenia)\footnote{GPs were offered a cash incentive for a specific number of vaccinations (\cite{baltic_news_network_estonia_2021}; \cite{slovenia_times_government_2021})}
and state employees
(Czechia)\footnote{Additional holiday \parencite{euronews_czech_2021}},
but these are not considered as a big enough shock to vaccination rates,
since larger parts of the public were not directly targeted. Estonia,
Czechia and Slovenia are therefore not removed. Lithuania offered a cash
incentive\footnote{see \textcite{lithuanian_national_radio_and_television_lithuanian_2021}}
to its citizens, but only in October (after the end of the lottery),
meaning that it does not have to be removed when restricting the
analysis to an end date of September 30, 2021. Bulgaria organised a
small raffle, giving out around 100 smartwatches to the vaccinated
\parencite{radio_bulgaria_bulgarias_2021}. Such a policy is not
comparable to a full scale lottery (Poland, Slovakia and Romania) and
should not have led to a large shock to the outcome variable, meaning
that Bulgaria stays in the donor pool.

Another important aspect in the choice of donor pool countries is the
availability of data. For most of the mentioned initial candidates,
several observations are missing. Sometimes these missing values follow
a specific pattern (e.g.~Poland: observations are missing on Sundays in
a specific time frame) while there is no specific pattern for other
countries. Linear interpolation is used to replace the missing
observations, by drawing a straight line between the two adjacent data
points. Other imputation techniques were also considered. Some of these,
such as last observation carried forward (LOCF) or mean imputation do
not seem attractive in the given setting, since the vaccination rate is
a monotonically increasing function.

It was not possible to obtain a reasonable trend for all of the
countries using linear interpolation, namely for Croatia and Hungary. As
a consequence, these two countries are also removed from the donor pool.
The donor pool therefore consists of six countries: Bulgaria, Czechia,
Estonia, Latvia, Lithuania and Slovenia.

\begin{table}[! htbp]\centering \caption{Predictors of Poland, Synthetic Poland and donor pool mean}
\bigskip 
\label{table:summarystat}
\begin{threeparttable}
\begin{tabular}{l c c c}
\toprule\midrule
 & \thead{Poland}
 & \thead{Synthetic Poland} & \thead{Mean donor}\\ \midrule
GDP per capita\tnote{a} & $12,810$ & $17,302.694$ & $14,203.333$ \\ 
Influenza vaccination rate\tnote{b} & $0.104$ & $0.170$ & $0.156$ \\ 
Population density\tnote{c} & $123.600$ & $89.509$ & $68.433$ \\ 
Share with tertiary education\tnote{d} & $0.289$ & $0.306$ & $0.310$ \\
Share of elderly\tnote{e} & $0.182$ & $0.203$ & $0.203$ \\ 
Trust in science\tnote{f} & $0.872$ & $0.884$ & $0.877$ \\ 
\bottomrule
\end{tabular}
\begin{tablenotes}\footnotesize
\item[a] in USD, 2020 (Eurostat)
\item[b] among elderly (over 64), 2019 (Eurostat)
\item[c] in persons per \(\text{km}^{2}\), 2019 (Eurostat)
\item[d] 15 to 64 years, 2020 (Eurostat)
\item[e] over 64 years, 2020 (Eurostat)
\item[f] 2020 (Wellcome Global Monitor)
\end{tablenotes}
\end{threeparttable}
\label{table2}
\end{table}

In order to improve the fit of the synthetic control unit, additional
predictors (\(\mathbf{Z}_{j}\)) are used. As seen in Section 3.1, the
choice of these variables is crucial in determining the optimal weights
of the synthetic control unit. GDP per capita, the share of elderly
(over 65), the share of people (15-64) with tertiary education and
population density are all relevant determinants of COVID-19 vaccine
uptake (\cite{viswanath_individual_2021};
\cite{walkowiak_predictors_2021}). Data on these variables is taken from
\textcite{eurostat_eurostat_2023}. Additionally, the share of elderly
vaccinated against Influenza as well as trust in science are possibly
additional indicators of vaccine openness. One would expect countries
with higher Influenza vaccination rates (before the spread of COVID-19)
to have higher COVID-19 vaccination rates, mainly due to a more
prevalent culture of individual health prevention, specifically
vaccines. A similar reasoning applies to trust in science. When citizens
generally place more confidence in scientists, one would expect them to
be more likely to get vaccinated \parencite{rozek_understanding_2021}.
Data on influenza vaccination rates is also extracted from
\textcite{eurostat_eurostat_2023}, while data from the Wellcome Global
Monitor, collected by \textcite{our_world_in_data_share_2020}, is
employed for the variable trust in science (share of people who answered
``a lot'' or ``some'' to the question ``How much do you trust
science?'').

\begin{table}[! htbp]\centering \caption[Weights of Synthetic Poland (share of the fully vaccinated)]{Slovenia takes up the largest weight of the synthetic control unit (share of the fully vaccinated)}
\bigskip
\label{table:weightssynth}
\begin{threeparttable}
\begin{tabular}{l c c c c c c c c c c}
\toprule\midrule
\thead{Country} & & & & & & & & & & \thead{Weight}\\ \midrule
Bulgaria & & & & & & & & & & 0.043 \\ 
Czechia & & & & & & & & & & 0.107 \\
Estonia & & & & & & & & & & 0.005 \\
Latvia & & & & & & & & & & 0.214 \\ 
Lithuania & & & & & & & & & & - \\ 
Slovenia & & & & & & & & & & 0.631 \\ 
\bottomrule\addlinespace[1ex]
\end{tabular}
\end{threeparttable}
\label{table2}
\end{table}

Another important aspect in the practical application of synthetic
control methodology is the time of intervention. Although the actual
lottery started on July 1, 2021, the time of announcement (May 25, 2021)
is chosen as the time of intervention, since the date of completing the
initial vaccination protocol (having received two doses) was not
relevant for entry into the lottery. If the lottery had an effect, it
would be expected to be observable from the time of announcement (plus a
potential lag, since there is a waiting time between first and second
dose).

In order to inspect the robustness of the synthetic control, two
robustness checks are carried out. Firstly, the date of intervention is
changed to the start of the lottery as well as backdated by one month,
to assess the overall robustness but also to specifically investigate
the presence of an anticipation effect. Secondly, a ``leave-one-out''
analysis is employed, with respect to the additional predictor variables
and the donor pool countries. Each country and predictor is left out of
the constrained minimization once, while holding all other
specifications constant.

The data analysis was carried out in R using the synth and SCtools
packages, generating Synthetic Poland. Table 3.1 shows descriptive
statistics of the predictors of Poland, Synthetic Poland and the donor
pool mean of the predictors. Table 3.2 presents the composition of
Synthetic Poland, for the analysis of the share of the fully vaccinated,
with the respective unit weights. The data and R scripts can be found in
the corresponding
\href{https://github.com/benediktstelter/bachelor_thesis.git}{GitHub repository \ExternalLink}\footnote{See appendix for further information}
\hspace{-0.1cm}.

\chapter{Results}

To begin with, a regression discontinuity design is employed to estimate
the effect of the lottery on daily vaccinations, using a cutoff of May
25, 2021 (announcement of the lottery).

\begin{figure}[h]
\caption[Regression discontinuity analysis of daily vaccinations]{Regression discontinuity analysis does not suggest increase in daily vaccinations around the day of announcemnt of the lottery}

\begin{center}\includegraphics{bachelor_thesis_files/figure-latex/unnamed-chunk-2-1} \end{center}
\end{figure}

Figure 3.1 presents the results of the analysis. By comparing the
intercepts at the cutoff, it can be observed that the lottery did not
increase daily vaccinations at the time of announcement, with the
difference in the limits being negative (point estimate: -39,788.34). As
the figure shows, the lottery was announced when daily vaccinations were
close to their all time high. A few weeks after the start of the
lottery, there was a strong decrease in daily vaccinations. In order to
make statements about the effect of the lottery on vaccination rates, a
further analysis is needed to evaluate this effect specifically.

\begin{figure}[h]
\caption[Synthetic control analysis of the Polish lottery (share of the fully vaccinated)]{The synthetic control analysis of the share of the population fully vaccinated against COVID-19 shows a decoupling between Poland and Synthetic Poland, as well as a subsequent narrowing of the gap}

\begin{center}\includegraphics{bachelor_thesis_files/figure-latex/unnamed-chunk-3-1} \end{center}



\begin{center}\includegraphics{bachelor_thesis_files/figure-latex/unnamed-chunk-3-2} \end{center}
\end{figure}

The synthetic control method is therefore applied, as described in
Section 3.2. Figure 4.2 plots the vaccination rate (fully vaccinated)
for both Poland and Synthetic Poland. The pre-treatment fit of the
synthetic control unit is not perfect, but relatively good, with a
maximum variation of around one percentage point from actual Poland.
Following a short lag after the intervention, a slow decoupling between
Poland and its synthetic control unit can be observed. This difference
continues to increase to a maximum of around five percentage points,
around the end of July/beginning of August. As time progresses, the gap
tends to decrease again, and by the end of September (end of the
lottery), the vaccination rates of Poland and synthetic Poland are
nearly back to the same level, with a remaining difference of around one
percentage point.

In order to answer whether the estimated effect of one percentage points
is statistically significant, it is necessary to employ the permutation
based inference techniques discussed in Section 4.1. Figure 4.3 presents
the placebo study. As observable, the magnitude of the effect of Poland
is not comparably high. In fact, the effect for Poland is the least
extreme of all of the selected units. This finding is also confirmed
quantitatively. Using the discussed test procedure, a \textit{p}-value
of 0.5714 is obtained, meaning that the observed effect is not
statistically significant. Therefore, the hypothesis that the lottery
had no effect in increasing vaccination rates cannot be rejected.

\begin{figure}[h]
\caption[Placebo plot of Poland and donor pool]{Estimated differences between actual and synthetic control units do not suggest a signficant effect. The synthetic control units of the donor pool countries are placebos (constructed as if there was a treatment).}

\begin{center}\includegraphics{bachelor_thesis_files/figure-latex/unnamed-chunk-4-1} \end{center}
\end{figure}

As introduced, a synthetic control analysis of the share of the
population with at least one dose of any COVID-19 vaccination is also
carried out. While the results show differing patterns compared to the
presented analysis (possibly suggesting a negative impact of the
lottery), there also seems to be no statistically significant effect,
with a \textit{p}-value of 0.1429. The corresponding figure can be found
in the appendix.

\subsection*{Robustness checks}

Next, the robustness of the synthetic control of the share of the fully
vaccinated is assessed. Firstly, the time of the intervention is
changed. As discussed earlier, there are two plausible intervention
points: The announcement and the start of the lottery, with the
announcement being the preferred option. In order to inspect how robust
the synthetic control unit is, July 1, the start of the lottery is now
used as the intervention point. As the upper panel of Figure 4.4 shows,
changing the time of intervention from May 25 to July 1, 2021 has a
considerable effect on the synthetic control unit (largest absolute
change of a single weight: 0.631 (Slovenia)), but although the direction
of estimated effect changes, this effect can also not be regarded as
significant. Importantly, a lag after the announcement of the lottery is
visible with both specifications.

\begin{figure}[h]
\caption[Robustness check: Time of intervention]{Changes in the time of intervention have considerable effects on the synthetic control unit, but the main conclusion remains unchanged}

\begin{center}\includegraphics{bachelor_thesis_files/figure-latex/unnamed-chunk-5-1} \end{center}



\begin{center}\includegraphics{bachelor_thesis_files/figure-latex/unnamed-chunk-5-2} \end{center}
\end{figure}

A second way of changing the time of intervention is backdating, meaning
that the effect is estimated using an arbitrary earlier intervention
time. If the synthetic control is robust, no drastic changes from the
baseline result in figure 4.1 will be expected. A new synthetic control
unit is therefore constructed, with a ``fictional'' intervention time
one month (April 25, 2021) prior to the actual announcement. As
observable in the lower panel of Figure 4.4, the changes are clearly
visible (largest absolute change of a single weight: 0.827 (Estonia)),
but the new synthetic control is still able to track the path of actual
Poland until the intervention occurs relatively well. Similar to the
first change of intervention time, the estimated effect is also not
significant.

Another possible robustness check is to leave out certain predictors or
countries. Figure 4.5 presents the result of a leave-one-out analysis of
synthetic Poland, leaving out all of the predictors once, while keeping
the others in, repeating the same with respect to donor pool countries
(solving the constrained minimization with only five countries). This
robustness check also shows considerable effects on the synthetic
control unit. In both the upper and the lower panel of the figure, large
differences (around ten percentage points) in the estimated effect
arise.

\begin{figure}[h]
\caption[Robustness check: Leave-one-out analysis]{Leave-one-out analysis with respect to predictors (upper panel) and donor pool countries (lower panel) generates large discrepancies in the estimated effect}

\begin{center}\includegraphics{bachelor_thesis_files/figure-latex/unnamed-chunk-6-1} \end{center}



\begin{center}\includegraphics{bachelor_thesis_files/figure-latex/unnamed-chunk-6-2} \end{center}
\end{figure}

\chapter{Discussion}

The results of the synthetic control analysis show that there was no
statistically significant increase in the vaccination rate of the fully
vaccinated. A similar analyses of the share of the population with at
least one dose confirmed this main result, while also showing very
different patterns. Additionally, the results of a regression
discontinuity design, applied to daily vaccinations around the
intervention date, do not suggest an increase in daily vaccinations
around the day of announcement.

Interpreting the presented results is not easy, since the synthetic
control becomes less reliable with increasing time after the
intervention. One possible interpretation of the results pictured in
Figure 4.2 might nonetheless be that people who would have gotten
vaccinated anyways decided to do this earlier, to take advantage of the
lottery. This could potentially explain the temporarily larger gap
(decoupling between the two units and subsequent narrowing of the gap)
between Poland and Synthetic Poland. So, while the lottery was not
successful in getting more people vaccinated, it could have possibly
induced people to get vaccinated earlier.

The robustness checks show that the main finding (no significant effect)
is robust with respect to the time of intervention, thereby confirming
that this application of the synthetic control method does not suffer
from an anticipation effect. But, since considerable differences in the
overall pattern of the synthetic control unit arise, the
interpretability of Figure 4.2 is further diminished.

When performing the leave-one-out analysis, larger discrepancies (ten
percentage points) in the estimated effect are observable, also with
respect to its direction. Given the relatively low number of donor pool
countries and predictor variables, this result is not particularly
surprising and underlines the importance of the specification of a
synthetic control analysis. Although making small adjustments to the
specification of the analysis might seem trivial, it potentially changes
the overall interpretation of the result quite considerably (direction
of the effect).

There are several limitations of this application of the synthetic
control method, restricting the internal validity of the results and the
derived interpretation.

Firstly, the composition of donor pool countries is not ideal, with a
low number of countries as well as considerable differences between the
control units. As a consequence, the number of possible combinations of
the weights is limited, e.g.~compared to a synthetic control analysis of
the lottery in Ohio (for example \textcite{barber_conditional_2022}),
where all other US States (minus states that implemented similar
policies) can in principle be selected for the donor pool. The low
number of donor pool countries is also problematic for the inference
procedure, since it is difficult to obtain a good fit of the synthetic
control unit for six different countries, potentially also limiting the
interpretation of the inference procedure: One of the placebo synthetic
controls estimates a negative effect of more than 20 percentage points
if there was a lottery in one of the donor pool countries. This seems to
be very unrealistic. Therefore, not only the interpretability but also
the main result itself (no significant effect), should be treated with
caution.

Ohio is also a good example regarding the similarity between the treated
unit and the donor pool units. Other US states are often very similar to
Ohio, both in terms of general characteristics as well as vaccination
rates \parencite{mathieu_global_2021}. Although the donor pool countries
in this thesis are selected to be not too different compared to
Poland\footnote{See table 3.1 for a comparison between Poland and the average of the donor pool},
cross-country differences are expected to be larger than cross-state
differences. In total, the low number of donor pool countries and a
restricted similarity between Poland and the donor pool limit the
validity of the synthetic control analysis, since it might not have been
possible to obtain an optimal fit of the synthetic control unit.

Secondly, the effect of the vaccine lottery might be too small to be
relevant for a synthetic control analysis. The effect on vaccination
rates that has been estimated for similar lotteries is often relatively
small (e.g.~\textcite{barber_conditional_2022}: 1.5\%). Poland's lottery
offers more prizes, but a lower main prize than some lotteries in North
America. It might therefore be conceivable that the effect of the given
policy in Poland is too small for a synthetic control analysis, since
the effect of the intervention might not be distinguishable from other
relatively small shocks to the vaccination rate (e.g.~a public marketing
campaign to get vaccinated).

Another possible limitation of this analysis are the imputed values. As
explained in Section 3.2, some countries in the donor pool did not
report their vaccination rates on all days, including the country of
interest (Poland). Although the results of the interpolation look
reasonable for all of the donor pool countries, this might still have a
negative effect on the credibility of the presented synthetic control,
since undesired changes in the chosen weights of the synthetic control
unit could have been caused.

At the same time, there are also requirements of the synthetic control
which this application should fulfill. Firstly, no signs of an
anticipation effect are observed, as discussed. Secondly, the
possibility of spillover effects: There is no plausible argument for the
presence of spillover effects. Since countries who adopted similar
incentive policies were removed, no country in the donor pool used the
Polish lottery as an example for similar action. Lastly, the number of
pre-intervention outcomes does not represent a problem. The outcome
variable is taken into account from February 1, 2021 until the day
before the announcement (May 24, 2021). Since daily data is used, a
total of 113 pre-intervention observations of the outcome variable are
used, meaning that the number of predictors in the optimization is high
(all 113 pre-intervention observations plus the additional predictor
variables).

To improve the internal validity, two additional predictor variables
were considered, but ultimately not selected. One of these was a
``political variable'', since differences in vaccine uptake exist across
party preferences. While this problem might be the most well-known in
the US \parencite{ruiz_predictors_2021}, it is also believed to be a
relevant predictor of vaccine uptake (even before COVID-19) in Europe
(\cite{schernhammer_correlates_2022}; \cite{kennedy_populist_2019}).
But, when using other European countries as a donor pool, it is
difficult to compare political beliefs across countries, mainly because
of large differences in party ideologies. One could use the vote share
for parties along the groups in the European Parliament, but this is not
unproblematic: There are large differences between parties within
certain
groups\footnote{For example, the Renew Europe group in the EP consists of liberal parties in a very broad sense, including both left of centre liberal parties and liberal-conservative parties.}
and the very wide political landscape in Europe may make differences in
attitudes toward the COVID-19 vaccine very hard to compare.

Another predictor which was considered is trust in government, with the
data also available from the Wellcome Global Monitor. Trust in
government is potentially highly volatile. For example, when an
unpopular government is replaced by a new government, this might
increase the trust into the public sector dramatically in a very short
time span. The problem of high volatility should have been especially
pronounced in a time of crisis like the COVID-19 pandemic, where large
variations in infection rates could have lead to quick changes in public
opinion. While there may also exist deep-rooted cross-country
differences in trust in government (e.g.~possibly higher distrust in
former soviet influenced countries compared to western countries
\parencite{costa-font_institutional_2021}), the presence of the
described volatility means that the variable trust in government is not
selected.

To summarize, it is important to note that the causal interpretation of
the given results should not be overstated, because of the outlined
threats to internal validity. As shown in Section 2.1, even when
investigating the same lottery (Ohio), studies can show differing
results, since the specification of the synthetic control analysis can
have large consequences on the estimated effect. Regarding the external
validity, this thesis presents only one case study of vaccine lotteries.
While the results do not suggest a significant effect of the lottery on
the vaccination rate, the applicability of these results on (at first
glance) comparable policies is limited, mainly because of differences in
the design of lotteries, initial vaccination rates as well as other
country-specific predictors.

\chapter{Conclusion}

As seen, nudging and the use of economic incentives can be successful in
inducing changes in individual health behaviour, e.g.~increasing
physical activity. Lotteries and other incentives may have also
contributed to increasing vaccination rates in the COVID-19 pandemic,
for example in Ohio, where several studies using the synthetic control
method found positive effects on vaccine uptake
(\cite{mallow_covid-19_2022}; \cite{brehm_ohio_2022};
\cite{barber_conditional_2022}; \cite{sehgal_impact_2021}).

In order to empirically assess the impact of vaccine lotteries on
vaccination rates in this thesis, a vaccine lottery implemented in
Poland, from July 1, 2021 to September 30, 2021 was examined. The
lottery consisted of cash and non-cash prizes of around 140 million zł.
To estimate the effect of this program on vaccination rates, the
synthetic control method was selected. Synthetic Poland was constructed
out of a donor pool of six other Eastern European countries.

The results of the main analysis show no signs of a statistically
significant (\textit{p}-value: 0.5714) increase in the vaccination rate
(fully vaccinated) compared to the hypothetical scenario without the
lottery. This main finding is also confirmed by a synthetic control
analysis of the rate of the population with at least one dose of any
COVID vaccination, as well as a regression discontinuity analysis of
daily vaccinations around the announcement date, with both analyses not
suggesting a significant increase. It can however be observed that
vaccination rates (fully vaccinated) increase in the short-run, thereby
possibly suggesting that people who would have gotten vaccinated anyways
may have chosen to do this earlier, as a result of the lottery.

\AtEndDocument{\pagebreak
\begin{appendices}
GitHub repository: \url{https://github.com/benediktstelter/bachelor_thesis}

\noindent All of the data of the covariates and the R scripts used to analyse the data can be found in folder "scripts\_data", along with some additional information in README.


\begin{figure}[h]
\caption[]{The analysis of the share of the population with at least one dose of any COVID vaccination shows different patterns, but also does not suggest a sginificant effect}

\begin{center}\includegraphics{bachelor_thesis_files/figure-latex/unnamed-chunk-7-1} \end{center}



\begin{center}\includegraphics{bachelor_thesis_files/figure-latex/unnamed-chunk-7-2} \end{center}
\end{figure}

\end{appendices}

\chapter*{Affidavit}
\thispagestyle{empty}



I affirm that this Bachelor thesis was written by myself without any unauthorised third-party support. All used references and resources are clearly indicated. All quotes and citations are properly referenced. This thesis was never presented in the past in the same or similar form to any examination board. 

\noindent I agree that my thesis may be subject to electronic plagiarism check. For this purpose an anonymous copy may be distributed and uploaded to
servers within and outside the University of Mannheim.

\vspace{2\baselineskip}

\noindent German version:\\
Ich versichere, dass ich die vorliegende Arbeit ohne Hilfe Dritter und ohne Benutzung anderer
als der angegebenen Quellen und Hilfsmittel angefertigt und die den benutzten Quellen
wörtlich oder inhaltlich entnommenen Stellen als solche kenntlich gemacht habe. Diese Arbeit
hat in gleicher oder ähnlicher Form noch keiner Prüfungsbehörde vorgelegen.

\noindent Ich bin damit einverstanden, dass meine Arbeit zum Zwecke eines Plagiatsabgleichs in
elektronischer Form anonymisiert versendet und gespeichert werden kann.

\vspace{4\baselineskip}
\begin{center}
\parbox{.8\textwidth}{Mannheim, 17/03/2023 \hfill Benedikt Stelter}
\end{center}


}


 
\backmatter


 
\renewcommand\refname{References}
\printbibliography[title=References]


\newpage
\newenvironment{appendices}
    {\chapter*{Appendix}
\renewcommand{\thetable}{A.\arabic{table}}
\renewcommand{\thefigure}{A.\arabic{figure}}
\setcounter{table}{0}
\setcounter{figure}{0}
    }






 
\end{document}
